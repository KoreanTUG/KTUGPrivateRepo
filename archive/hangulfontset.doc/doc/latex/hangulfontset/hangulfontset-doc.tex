\documentclass[b5paper,nanum]{oblivoir}

\usepackage{fapapersize}
\usefapapersize{*,*,1in,*,1in,*}

\protected\def\cs#1{\texttt{\textbackslash #1}}
\protected\def\ct#1{\texttt{#1}}

\usepackage{kotex-logo}

\begin{document}

\title{한글 폰트 설정, hangulfontset}
\author{Nova De Hi}
\date{2014/06/09}

\maketitle

\hologo{XeTeX}, \hologo{LuaTeX}이 도입된 이래 한글 폰트를 맘대로 쓸 수 있게 되어서 좋습니다.
그런데 의외로 \cs{setmain...} 어쩌구 하는 한글 폰트 설정 명령에 애를 먹는 분들이 계신가봐요.
사실 어렵지는 않은데 귀찮다고 할 수 있겠죠.

간단한 패키지를 하나 만들었습니다. 이름하여 hangulfontset.sty.

당연히 트루타입, 오픈타입 글꼴이 설치되어 있어야 합니다. 특히 \ct{[barun]} 옵션을 위해서 나눔바른고딕이 설치되어
있어야 합니다.

\section{나눔글꼴 설정}

\begin{itemize}\tightlist
\item \cs{usepackage\{hangulfontset\}} 나눔명조/나눔고딕 한글이 설정됩니다.
\item \cs{usepackage[default]\{hangulfontset\}} 옵션 없이 쓰는 경우와 같습니다.
\item \cs{usepackage[defaultx]\{hangulfontset\}} 명조체 한글이 ExtraBold가 됩니다.
\item \cs{usepackage[barun]\{hangulfontset\}} 고딕체가 나눔바른고딕이 됩니다.
\item \cs{usepackage[barunx]\{hangulfontset\}} 고딕체를 나눔바른고딕으로 하고 명조체 볼드가 ExtraBold.
\end{itemize}

여기까지는 명조체에 한자가 나오지 않습니다.

\begin{itemize}
\item \cs{usepackage[nanum]\{hangulfontset\}} ExtraBold 명조와 한자. 한자는 NanumGothic의 것을 찍습니다.
\end{itemize}

\section{사전설정된 글꼴 집합}

\begin{itemize}
\item \cs{usepackage[hcr]\{hangulfontset\}} 함초롬LVT.
\end{itemize}

\section{한글 글꼴 설정하기}

\begin{description} \tightlist
\item [auto] 옵션: 볼드체를 자동으로 탐색합니다. 당연히 글꼴 가족이 정의되지 않은 글꼴에서는 찾기에 실패하므로 볼드도 그냥 보통체로 찍힙니다.
\item [user] 옵션; 볼드를 지정하면 그 글꼴을 쓰고 볼드의 지정이 없으면 FakeBold합니다. FakeBold가 xetex에서만 되는 건 다 아실테고요. 특히 이 옵션은 bold까지 모두 사용자가 지정해줄 때 씁니다.
\item [mj, mjbold, gt, gtbold] 각각의 글꼴 이름을 써넣습니다. 되도록 포스트스크립트 이름을 쓰는 것을 추천합니다. 예를 들면
\begin{verbatim}
\usepackage
 [user,mj={HCRBatangLVT},mjbold={HCRBatangLVT-Bold},
  gt={HCRDotumLVT},gtbold={HCRDotumLVT-Bold}]
 {hangulfontset}
\end{verbatim}
\end{description}
auto나 user를 쓸 때, mj만 지정하면 gt 폰트도 이 폰트 설정을 따라갑니다. 반면 gt만 설정하는 것은 mj에 영향을 주지 않습니다.

\section{한자 글꼴}

한자 글꼴을 별도로 지정하려면 \ct{[hanja]} 옵션을 활성화하거나 \ct{[hanjabyhangul]}을 주어야 합니다.
\ct{[hanja]}를 선언한 후에는 \ct{hjmj, hjmjbold, hjgt, hjgtbold}를 지정할 수 있으며, \ct{auto} 옵션이 주어지면 \ct{hjmj}로 볼드체를 자동으로 찾으려 합니다. \ct{user} 옵션이 주어지면 \ct{hjmjbold}가 있는 것이 좋고 없으면 FakeBold합니다.

\ct{[hanjabyhangul]}은 되도록 한글 글골에서 한자를 찾으려합니다만 한자가 없는 글꼴이라면 도리없겠죠.

\section{속성의 추가}

\begin{itemize}\tightlist
\item 일단 이 패키지로 글꼴을 설정하면 FakeSlant가 동작합니다. (BoldItalic은 설정하지 않았습니다. 이게 필요할까요? 의견이 있으시면 반영하고요.)

\item \ct{Ligatures=TeX}을 기본적으로 넘겨줍니다.

\item 이 이외에 추가적인 font feature를 지정하려면, feature 옵션을 이용합니다.
\begin{verbatim}
\usepackage[feature={Color=blue}]{hangulfontset}
\end{verbatim}

\item mjfeature와 gtfeature로 따로 구분하여 지정할 수도 있습니다. WordSpace라든가, interhchar라든가... 필요한 게 있으면 이 옵션을 이용하세요.
\end{itemize}

\section{제한}
\begin{itemize} \tightlist
\item \hologo{XeTeX}, \hologo{LuaTeX}에서만 동작합니다.
\item \koTeX (\hologo{XeTeX}-\ko, \hologo{LuaTeX}-\ko)의 한글 폰트 설정 방식을 전제로 합니다.
\item \cs{fallbackhanjafont}나 \cs{fallbackhangulfont} 등 fallback 폰트관련 사항은 이 패키지에서 지원하지 않습니다.
\item \cs{setmonohangulfont} 계열도 지원하지 않습니다. 필요하면 직접 써넣으세요.
\end{itemize}

\section{예제}

hcr을 기본 한글 폰트로 하고 한자 폰트만 mj/gt로 각각 부여해본 경우입니다.
\begin{verbatim}
    \usepackage
        [hcr,hanja,hjmj={DX해서M.ttf},hjgt={MalgunGothic}]
        {hangulfontset}
\end{verbatim}

\noindent KoPub으로 설정해본 예입니다.
\begin{verbatim}
    \usepackage[user,  %% KoPub은 bold체를 지동으로 못 찾습니다.
       mj={KoPubBatangLight},
       mjbold={KoPubBatangBold},
       gt={KoPubDotumMedium},
       gtbold={KoPubDotumBold},
       hanjabyhangul,
       mjfeature={WordSpace=1.2,interhchar=-.03em},
       gtfeature={WordSpace=1.1}]
       {hangulfontset}
\end{verbatim}

\section{기타}
사용해보시고 이상동작, 기능개선 등이 필요하면 답글을 달아주세요.\footnote{%
	\url{http://www.ktug.org/xe/index.php?document_srl=183374}
}

\end{document}
