\documentclass[figtabcapt,a4paper]{oblivoir}

\usepackage{fapapersize}
\usefapapersize{*,*,1in,*,1in,*}

\usepackage[most]{tcolorbox}

\setmainfont{Minion Pro}
\setsansfont{Myriad Pro}

\usepackage[badukpansize=8,badukpancolor=cyan!10,textcmds]{ksbaduk}
\NumberFont{\sffamily\fontsize{8}{8}\selectfont}
%\BadukpanSize{10}

%\usepackage{ksforloop}

\def\bs{\textbackslash}

\begin{document}

\title{ksbaduk: 기보 조판, version 0.6.3}
\author{Nova De Hi}
\maketitle

\tableofcontents*

\listoffigures*

\section{도입}

상세한 것은 \url{http://www.ktug.org/xe/index.php?document_srl=205706&mid=KTUG_open_board}.

이 패키지를 만드는 동기를 제공하신 `잠/하' 님, 그로몹 님, 그리고 이름을 우리말 식으로 할 것을 제안해주신 작나 님, 설명서의 영문판을 만드는 데 참여하신 그로몹, ischo, Hoze께 감사드린다.

\section{사용법}

\subsection{패키지 옵션}

\begin{boxedverbatim}
\usepackage[%
    ball, % plain
    badukpancolor=<color>,
    badukpansize=<number>,
    posmark,
    imageback,
    imagefile=<file>,
    textcmds,
    numberfont=<font>,
    tmarkfont=<font>
]{ksbaduk}
\end{boxedverbatim}

\begin{description}
\item [\texttt{ball}, \texttt{plain}] 바둑돌에 음영을 넣거나 넣지 않는다. \texttt{plain}이 디폴트.
\item [\texttt{badukpancolor}] 바둑판의 색상을 지정한다. \textsf{xcolor} 색상명으로 쓴다. 기본값은 \texttt{yellow!20}.
\item [\texttt{badukpansize}] 바둑판의 크기를 지정한다. 양수 값으로 준다. 기본값은 \texttt{10}. 이 값이 변하면 수순을 나타내는 숫자나 바둑판 위에 오는 텍스트와 잘 맞지 않을 경우도 생기므로 이 때는 별도로 해당 값들을 지정해야 할 수 있다. 지나치게 크거나 작게 확대/축소하지 않는 것이 좋다.
\item [\texttt{posmark}] 바둑판의 왼쪽과 상단에 좌표 이름을 표시한다. 단, 이 옵션이 주어지더라도 바둑판 전체가 아니라 일부만 보이는 환경에서는 표시되지 않는다.
\item [\texttt{imageback}] 바둑판의 배경에 이미지를 넣는다.
\item [\texttt{imagefile}] 배경으로 사용할 이미지 파일을 지정한다. 같은 디렉터리나 찾을 수 있는 디렉터리에 해당 파일이 있어야 한다. 디폴트로 \texttt{badukpan.jpg}를 사용한다.
\item [\texttt{textcmds}] 텍스트 영역에 바둑알 모양을 그려야 할 때가 있다. 디폴트를 이 명령을 사용하지 않는 것이다. 텍스트 명령을 사용하려면 명시적으로 이 옵션을 활성화하여야 한다.
\item [\texttt{numberfont}] 수순 번호를 식자할 폰트를 지정한다. 바둑판을 너무 작은 크기로 줄이거나 키우더라도 숫자가 자동으로 줄어들지 않을 수 있다. 기본값이 적절하지 않으면 이 옵션으로 폰트 명령을 지정한다.
\item [\texttt{tmarkfont}] \verb|\TextMark|나 \verb|\Blanket| 명령으로 식자되는 문자에 적용될 폰트를 지정한다.
\end{description}

\subsection{설정 명령}
다음 명령들이 제공된다.

\begin{description}
\item [\texttt{\bs BadukpanSize}] 옵션의 \verb|badukpansize|와 같은 역할을 한다.
\item [\texttt{\bs BackgroundColor}] 옵션의 \verb|badukpancolor|와 같은 역할을 한다. 문서 중간에 변경할 때 사용할 수 있다. \texttt{\bs BadukpanColor} 형식으로 써도 된다.
\item [\texttt{\bs NumberFont}] 옵션의 \verb|numberfont|와 동일하다.
\end{description}

이 명령들은 preamble이나 문서 중간을 막론하고 사용할 수 있다.

\subsection{환경}

\begin{boxedverbatim}
\begin{ksbadukpan}[<options>][<pos>]
...
\end{ksbadukpan}
\end{boxedverbatim}

이 환경은 두 개의 옵션 인자를 가진다. 첫째 것은 패키지의 옵션 인자와 같다. 단 \texttt{textcmds}는
개별 환경에서 의미가 없으므로 제외. 여기에 올 수 있는 옵션을 열거하면 다음과 같다.
\begin{verbatim}
[ball,badukpancolor=<color>,badukpansize=<size>,%
 posmark,imageback,imagefile=<file>,numberfont=<font>,tmarkfont=<font>]
\end{verbatim}

둘째 것은 바둑판 전체를 보이지 않고 일부만 보이고자 할 때 쓸 수 있다. 여기에 올 수 있는 것은
\texttt{U, D, L, R, UL, UR, DL, DR} 중 하나인데, 각각 상변, 하변, 좌변, 우변,
좌상귀, 우상귀, 좌하귀, 우하귀 부분만 일부 보여주는 역할을 한다. 이 옵션이 없으면 전체 바둑판을 모두 그린다.

이 환경 대신 다음 명령을 사용할 수 있다.
\begin{verbatim}
\StartBaduk
\StartBadukClip{U}

\StopBaduk
\end{verbatim}
\verb|\StartBaduk| 또는 \verb|\StartBadukClip|은 반드시 \verb|\StopBaduk|로 끝나야 하며 이것은
위의 \verb|ksbadukpan| 환경을 쓴 것과 동일하다. 이 환경 영역 내에 일반 텍스트는 올 수 없다.

\subsection{좌표}

\begin{figure}[h]
\centering
\begin{ksbadukpan}[posmark,imageback,imagefile=badukpan.jpg]
\end{ksbadukpan}
\caption{좌표}\label{fig:badukpan}
\end{figure}

위치를 나타내는 좌표는 알파벳 A에서 T까지(단 I는 J와 같음), 숫자 1에서 19까지를 잇대어 쓴다. 열(column)이 알파벳 문자이고 행(row)이 숫자이다. \fref{fig:badukpan}\를 보라.
좌표를 표시하기 위해 알파벳 대문자나 소문자 모두 사용이 가능하지만 history 등에서 대문자와 소문자는 서로 다른 위치로 기록되므로 예를 들어 \texttt{D4}로 표시된 기록을 \texttt{d4}로 제거할 수 없다.
일관성있게 사용하는 것이 좋다.

\subsection{착점 그리기: 수순표시 없이}

흑돌 또는 백돌을 특정 위치에 착점하려면 다음과 같이 한다.
\begin{boxedverbatim}
\Black{D4} \White{F3}
\end{boxedverbatim}

\begin{ksbadukpan}[badukpansize=7,badukpancolor=yellow!20][DL]
\Black{D4} \White{F3}
\end{ksbadukpan}

\begin{boxedverbatim}
\Blacks{D4,F3,C7} \Whites{D4,F3,C7}
\end{boxedverbatim}

\begin{ksbadukpan}[badukpansize=7,badukpancolor=yellow!20][DL]
\Blacks{D4,F3,C7}
\end{ksbadukpan}
\quad
\begin{ksbadukpan}[badukpansize=7,badukpancolor=yellow!20][DL]
\Whites{D4,F3,C7}
\end{ksbadukpan}

\verb|\Blacks|와 \verb|\Whites|는 여러 개의 같은 돌을 식자하게 한다. 각 위치는
쉼표(\verb|,|)로 분리하여 입력한다.


\begin{boxedverbatim}
\BlackFirst{D4,F3,C7}
\end{boxedverbatim}

\begin{ksbadukpan}[badukpansize=7,badukpancolor=yellow!20][DL]
\BlackFirst{D4,F3,C7}
\end{ksbadukpan}

흑선으로 차례로 착점한 결과를 표시한다. 백선을 의미하는 \verb|\WhiteFirst|가 있다.

\subsection{수순 표시}

\begin{boxedverbatim}
\BlackN{D4}{1}
\WhiteN{F3}{2}
\end{boxedverbatim}

\begin{ksbadukpan}[badukpansize=7,badukpancolor=yellow!20][DL]
\BlackN{D4}{1}
\WhiteN{F3}{2}
\end{ksbadukpan}

이 명령은 반드시 두 개의 인자를 취한다. 첫째 인자는 좌표이고 둘째 인자는 표시할 수순번호이다.

\begin{boxedverbatim}
\BlackFirstN{D4,F3,C7}[5]
\end{boxedverbatim}

\begin{ksbadukpan}[badukpansize=7,badukpancolor=yellow!20][DL]
\BlackFirstN{D4,F3,C7}[5]
\end{ksbadukpan}

흑선으로 5부터 시작하는 수순을 표시하였다. 대응하는 백선 명령은 \verb|\WhiteFirstN|이다.

옵션 인자가 생략되면 1부터 시작한다.

\begin{boxedverbatim}
\BlackFirstN{D4,F3,C7}
\end{boxedverbatim}

\begin{ksbadukpan}[badukpansize=7,badukpancolor=yellow!20][DL]
\BlackFirstN{D4,F3,C7}
\end{ksbadukpan}

옵션 인자에 별표(\verb|*|)를 쓰면 현재까지 진행된 수순번호에 이어서 나타낸다.
이 시작 수순번호의 자동 지정은 \verb|\BlackFirstN|이나 \verb|\WhiteFirstN|으로 입력된
돌에 한하여 계산한다.

\begin{boxedverbatim}
\BlackFirstN{r16,p16,n17}
\WhiteFirstN{r17,q17,q16}[*]
\BlackFirstN{p17,o17,o16,o18,r18,s17}[*]
\end{boxedverbatim}

\begin{ksbadukpan}[badukpansize=7,badukpancolor=yellow!20,posmark][UR]
\BlackFirstN{r16,p16,n17}
\WhiteFirstN{r17,q17,q16}[*]
\BlackFirstN{p17,o17,o16,o18,r18,s17}[*]
\end{ksbadukpan}

이 명령에 별표 옵션(\verb|*|)을 붙이면 수순번호를 표시하지 않게 된다. 즉 \verb|\BlackFirst|와 같은 결과가 된다. 
%이 옵션은 특히 이전 설명도로부터 다음 장면을 설명하거나 할 때 이미 입력된 착점의 수순번호를 간단히 가릴 수 있게 하여 입력의 수고를 던다.

\begin{boxedverbatim}
\BlackFirstN*{D4,F3,C7}
\end{boxedverbatim}

\begin{ksbadukpan}[badukpansize=7,badukpancolor=yellow!20][DL]
\BlackFirstN*{D4,F3,C7}
\end{ksbadukpan}

\subsection{표지(mark)}

반상의 수를 참조하거나 설명하기 위하여 표지를 붙인다. 이를 위한 명령이 마련되어 있다.

\begin{boxedverbatim}
\BlackM{C7} \WhiteM{D7} \WhiteC{F7} \BlackC{G7}
\BlackD{C8} \WhiteD{D8} 
\BlackDs{D4,F3} \WhiteMs{B2,D3} \BlackCs{B5,C5}
\end{boxedverbatim}

\begin{ksbadukpan}[badukpansize=7,badukpancolor=yellow!20][DL]
\BlackM{C7} \WhiteM{D7} \WhiteC{F7} \BlackC{G7}
\BlackD{C8} \WhiteD{D8} 
\BlackDs{D4,F3} \WhiteMs{B2,D3} \BlackCs{B5,C5}
\end{ksbadukpan}

M은 Mark이고 C는 Circle, D는 Dead이다. 각각 둘 이상을 적을 때에는 \texttt{s}가 덧붙은 명령으로 한다.

수순 해설에서 가끔 필요한 특정 위치의 부호 문자를 식자하려면 다음 명령을 쓴다. 아래 예시한 두 명령은
의미가 같다. 참고로, 이 \verb|\TextMark| 명령으로 식자되는 문자의 폰트를 조절하려면 환경 옵션 인자로
\texttt{tmarkfont=<font>} 옵션을 부여하면 된다.
\begin{boxedverbatim}
\Blanket{D4}{가}  \TextMark{D7}{A}
\end{boxedverbatim}

\begin{boxedverbatim}
\begin{ksbadukpan}[tmarkfont={\sffamily\bfseries\normalsize}][DL]
\TextMark{D7}{A}
\end{ksbadukpan}
\end{boxedverbatim}

\begin{ksbadukpan}[badukpansize=7,badukpancolor=yellow!20][DL]
\Blanket{D4}{가}  \TextMark{D7}{A}
\end{ksbadukpan}
\quad
\begin{ksbadukpan}[badukpansize=7,badukpancolor=yellow!20,tmarkfont={\sffamily\bfseries\normalsize}][DL]
\TextMark{D7}{A}
\end{ksbadukpan}


\subsection{이전까지의 수순 history}

\begin{boxedverbatim}
\KSBadukContinue
\end{boxedverbatim}

이 명령은 현재 문서에서 지금까지 조판한 모든 기록을 가지고 있다가 모두 (수순번호 없이) 출력한다.
한 국을 여러 장면으로 나누어 표기할 때 유용한 명령이다.
이 명령에 별표(\verb|*|)를 붙이면 이전의 history를 백선으로 식자한다.\footnote{%
	만약 현재 환경이 일부를 표시하는 상황이고 history 중 일부가 가려지는 경우에
	가끔 흑/백이 바뀌어 나타날 수 있다. 이 때 별표붙은 명령을 쓸 수 있다.}

\begin{boxedverbatim}
\ClearHistory
\end{boxedverbatim}

보관하고 있던 모든 조판 기록을 지우고 초기화한다.

\subsection{따낸 돌과 참고도}

\begin{boxedverbatim}
\RemoveStone{<pos>}
\end{boxedverbatim}
이 명령은 따낸 돌을 history에서 제거하는 역할을 한다.
이 명령은 여러 개를 쉼표로 구분하여 지울 수 있다.
따낸 돌은 다음과 같이 하여 다음 장면도에 표시되지 않게 한다.

\ClearHistory

\begin{boxedverbatim}
\begin{ksbadukpan}[badukpansize=6][D]
\BlackFirstN{K5,L5,L6,K6,J6,L7,K7}
\RemoveStone{K6}
\end{ksbadukpan}
\begin{ksbadukpan}[badukpansize=6][D]
\KSBadukContinue
\end{ksbadukpan}
\end{boxedverbatim}

\begin{ksbadukpan}[badukpansize=6][D]
\BlackFirstN{K5,L5,L6,K6,J6,L7,K7}
\RemoveStone{K6}
\end{ksbadukpan}
\hfill
\begin{ksbadukpan}[badukpansize=6][D]
\KSBadukContinue
\end{ksbadukpan}

흑이나 백의 수순 중에서 해당 돌만을 제거하려 할 때는 옵션 인자로 \texttt{B} 또는 \texttt{W}를 지정할 수 있다.
\begin{verbatim}
\RemoveStone[W]{K6}
\end{verbatim}
이 인자가 지정되지 않으면 해당 위치의 돌을 (흑/백을 불문하고) 제거한다.

참고도는 실전 진행과 상관없이 제시되는 것이다. 참고도에서 착수한 점들은 다음 진행도를
보일 때 표시되면 안 된다. 이 경우에도 \verb|\RemoveStone|한다.

\ClearHistory

\noindent
\begin{boxedverbatim}
\begin{ksbadukpan}[badukpansize=6][U]
\BlackFirst{r16,p16,n17}
\WhiteFirstN{r17,q17,q16,p17,o17,o16,o18,r18,s17}
\RemoveStone{o17,o16,o18,r18,s17}
\end{ksbadukpan}
\hfill
\begin{ksbadukpan}[badukpansize=6][U]
\KSBadukContinue
\end{ksbadukpan}
\end{boxedverbatim}

\begin{ksbadukpan}[badukpansize=6][U]
\BlackFirst{r16,p16,n17}
\WhiteFirstN{r17,q17,q16,p17,o17,o16,o18,r18,s17}
\RemoveStone{o17,o16,o18,r18,s17}
\end{ksbadukpan}
\hfill
\begin{ksbadukpan}[badukpansize=6][U]
\KSBadukContinue
\end{ksbadukpan}

왼쪽 그림의 5, 6, 7, 8, 9 다섯 수는 오른쪽 그림에 나타나지 않는다.

\ClearHistory

\subsection{장면도}

한 국의 바둑을 여러 장면으로 나누어 표시하는 때가 많다. \texttt{ksbadukpan} 환경을 벗어나지
않고 약간의 텍스트를 끼워넣거나 다음 장면을 그리기 위한 명령이 준비되어 있다.

\begin{boxedverbatim}
\ProceedNextScene{<cmds>}
\end{boxedverbatim}

이 명령이 주어진 시점에서 다음 장면도를 그리기 시작한다.
인자는 줄 수도 있고 아니 줄 수도 있는데 앞선 바둑판과 다음 바둑판 사이의 거리를 조절하거나 하기 위한 명령들이
올 수 있다. 단, 줄바꿈을 위해서는 \verb|\par|가 아니라 \verb|\KSpar|를 써야한다는 점에 주의한다.
이전까지의 진행은 다음 그림에서 번호없이 그려진다.

\begin{boxedverbatim}
\ProceedNextSceneComment{문장}
\end{boxedverbatim}

앞의 \verb|\ProceedNextScene|과 같지만 인자로 주어진 문장을 그 사이에 식자한다.
길지 않은 해설이 들어갈 때 사용할 수 있을 것이다.

\begin{boxedverbatim}
\KSpar
\end{boxedverbatim}

위의 두 명령에서 문단 나누기를 해야 할 필요가 있을 때 \verb|\par|는 쓸 수 없다. 그 대신
이 명령을 사용한다. 이 명령들은 항상 \texttt{ksbadukpan} 환경 내부에서 쓰여야 한다.

위에 제시된 세 가지 명령이 모두 사용된 예를 보자.

\begin{verbatim}
\begin{ksbadukpan}[badukpansize=6,badukpancolor=yellow!20]
\BlackFirstN{R16, Q4, D3, O16, P17, D6, F4, D17, P16, O15}
\ProceedNextSceneComment{\KSpar 다음 착점은 어디? \KSpar\medskip}
\BlackFirstN{P15, O14, O17, C14, J17, G17, L17}[*]
\ProceedNextScene{\quad}
\WhiteFirstN{C4, C3, C9, P14, R10, O13, R7}[*]
\end{verbatim}

\begin{ksbadukpan}[badukpansize=6,badukpancolor=yellow!20]
\BlackFirstN{R16, Q4, D3, O16, P17, D6, F4, D17, P16, O15}
\ProceedNextSceneComment{\KSpar 다음 착점은 어디? \KSpar\medskip}
\BlackFirstN{P15, O14, O17, C14, J17, G17, L17}[*]
\ProceedNextScene{\quad}
\WhiteFirstN{C4, C3, C9, P14, R10, O13, R7}[*]
\end{ksbadukpan}

\bigskip

이 명령은 예를 들어 수백 수가 이어지는 한 국 전체를 조판하기에 적당하지 않다.
간단히 몇 개의 장면으로 나누어 조판하는 편의를 위한 것이므로 남용하는 것을 권하지 않는다.
또한, 장면도를 figure 환경에 넣는 경우에도 이 명령으로 조판하는 것은 적절치 않을 수 있다.


\subsection{게임의 저장과 복구}

현재까지 진행된 수순을 저장해두었다가 예를 들면 참고도 등으로 진행한 후라도
다시 저장된 위치의 바둑판 상황을 복구하도록 하였다. 이 아이디어는 그로몹 님께서 제안하신 것이다.

\begin{boxedverbatim}
\SaveKSBaduk{<unique name>}
\LoadKSBaduk{<saved name>}
\end{boxedverbatim}

저장하거나 불러올 때는 반드시 이름을 붙여야 한다. 이 이름은 \TeX 이 명령에서 사용하는 letter(=11) 카테고리에 속하는 문자를 쓴다. 숫자는 쓸 수 없다. 

\begin{boxedverbatim}
\DeleteSavedKSBaduk{<name>}
\end{boxedverbatim}

특별한 경우, 저장된 게임을 지워야 할 때가 있다. 

다음 그림에서 첫 번째 장면도가 끝난 시점에서 저장하고 두 번째 진행 후에 저장된 것을 불러오는 것을 보인다.

\begin{boxedverbatim}
\begin{ksbadukpan}[badukpansize=6.5]
\BlackFirstN{c4,q3,r16,e16}
\BlackFirstN{p17,e3,q5,q9,r11,p4}[*]
\SaveKSBaduk{scenei}  %% 여기서 저장
\end{ksbadukpan}

\begin{ksbadukpan}[badukpansize=6.5]
\KSBadukContinue
\BlackFirstN{j3,d6,e4,d4,d5,d3}[*]
\end{ksbadukpan}

\begin{ksbadukpan}[badukpansize=6.5]
\LoadKSBaduk{scenei}  %% 여기서 복구
\end{ksbadukpan}
\end{boxedverbatim}

\bigskip

\ClearHistory

\begin{ksbadukpan}[badukpansize=6.5]
%% 1-4
\BlackFirstN{c4,q3,r16,e16}
%% 5-10
\BlackFirstN{p17,e3,q5,q9,r11,p4}[*]
\SaveKSBaduk{scenei}
\end{ksbadukpan}
\hfill
\begin{ksbadukpan}[badukpansize=6.5]
\KSBadukContinue
%% 11-16
\BlackFirstN{j3,d6,e4,d4,d5,d3}[*]
\end{ksbadukpan}

\begin{ksbadukpan}[badukpansize=6.5]
\LoadKSBaduk{scenei}
\end{ksbadukpan}

\subsection{\texttt{sgf} 형식의 입력}

sgf (Smart Game Format)\footnote{\url{http://en.wikipedia.org/wiki/Smart_Game_Format}}는 표준 기보 포맷이다.
sgf 포맷에서 경기 진행 기록 부분을 일괄 식자할 수 있다.
유의할 사항은 다음과 같다.
\begin{enumerate} \firmlist
\item sgf의 metadata를 처리하지 못한다.
\item 주석문(\texttt{C})을 처리하지 못한다. 그러므로 다음 명령 안에 \texttt{C}문이 있으면 안 된다.
\item 반상에서 죽은 돌을 자동으로 제거하지 않는다. 이것은 적당한 곳에 진행을 나누어 입력하면서 입력자가 \verb|\RemoveStone|해주어야 한다. 이 때 돌의 위치는 이 패키지가 정의한 좌표를 써야 한다.
\item \verb|;| 구분자에 민감하다. 즉 \verb|;B|와 \verb|;W|만을 정상 인식한다. 그 이외의 경우에 \verb|;|가 나오면 안 된다.
\end{enumerate}

\begin{boxedverbatim}
\SGFLine{;B[pd];W[dc];B[pq];W[dp];B[de];W[dh]}
\SGFLine{;B[gd];W[ee];B[ef];W[fe];B[cc];W[fc]}
\end{boxedverbatim}

\begin{ksbadukpan}[badukpansize=6]
\SGFLine{;B[pd];W[dc];B[pq];W[dp];B[de];W[dh]}
\SGFLine{;B[gd];W[ee];B[ef];W[fe];B[cc];W[fc]}
\end{ksbadukpan}

이 명령은 문서 전체에 대하여 수순번호가 이어지게 되어 있다. 이것을 리셋하려면
\begin{boxedverbatim}
\ResetSGFCounter
\end{boxedverbatim}
이 명령의 옵션 인자로 직전 수 카운터를 부여할 수 있다.
예를 들어 다음에 착점할 수가 13으로 시작하게 하려면 `12'를 지정한다. 옵션 인자가 없으면
`0'으로 초기화되며, 그 다음 첫 수를 `1'로 식자할 것이다.

\begin{boxedverbatim}
\ResetSGFCounter[12]
\SGFLine{;B[db];W[df];B[ce];W[eb];B[dd];W[ec]}
\end{boxedverbatim}

\begin{ksbadukpan}[badukpansize=6][UL]
\KSBadukContinue
\ResetSGFCounter[12]
\SGFLine{;B[db];W[df];B[ce];W[eb];B[dd];W[ec]}
\end{ksbadukpan}


\subsection{텍스트 명령들}

아래의 명령들을 사용하려면,
\begin{verbatim}
\usepackage[textcmds]{ksbaduk}
\end{verbatim}
이와 같이 패키지 옵션으로 textcmds를 주어야 한다.

\begin{boxedverbatim}
\WhiteNText{<num>}, \BlackNText{<num>},
\WhiteMText, \BlackMText,
\WhiteCText, \BlackCText
\end{boxedverbatim}

위의 명령들은 \texttt{ksbadukpan} 환경 안에서 쓰이면 안 된다. 보통의 텍스트 식자 상황에서 사용한다.
위의 결과는 다음과 같다.

\WhiteNText{10}, \BlackNText{25}, \WhiteMText, \BlackMText,
\WhiteCText, \BlackCText

바둑판 환경의 난외에 따낸 돌을 표시해야 할 경우에 이 명령을 쓸 수 있다.

\section{`잠하' 님의 병국세해 조판}

\renewcommand\figurename{장면도}
\setcounter{figure}{0}

다음 몇 예제는 이 패키지를 만드는 계기가 된 `잠하' 님이 조판하신 병국세해 문서에 나오는 것이다.
현재 pdf 문서에서 소스는 보이지 않으려 했는데 빈 공간이 생겨서 처음 두 장면도의 소스를 함께 보이겠다.

\ClearHistory

\begin{figure}[!h]
\centering
\caption{프로들은 안 두는 수}
\begin{ksbadukpan}[posmark,imageback]
\BlackFirstN{Q4,D16,D4,R16,P16,M16,P14,R14}
\end{ksbadukpan}
\end{figure}

\begin{boxedverbatim}
\renewcommand\figurename{장면도}

\begin{figure}[!h]
\centering
\caption{프로들은 안 두는 수}
\begin{ksbadukpan}[posmark,imageback]
\BlackFirstN{Q4,D16,D4,R16,P16,M16,P14,R14}
\end{ksbadukpan}
\end{figure}
\end{boxedverbatim}

\begin{figure}[!h]
\centering
\caption{가능은 하다}
\begin{ksbadukpan}[posmark]
\KSBadukContinue
\BlackFirstN{P12,R12,R17,S17,Q17,J16}
\Blanket{R18}{A}
\Blanket{K14}{B}
\end{ksbadukpan}
\end{figure}

\begin{boxedverbatim}
\begin{figure}[!h]
\centering
\caption{가능은 하다}
\begin{ksbadukpan}[posmark]
\KSBadukContinue
\BlackFirstN{P12,R12,R17,S17,Q17,J16}
\Blanket{R18}{A}
\Blanket{K14}{B}
\end{ksbadukpan}
\end{figure}
\end{boxedverbatim}

\begin{figure}[!h]
\centering
\begin{minipage}{.48\textwidth}
\centering
\caption{최초의 의문수}
\begin{ksbadukpan}[badukpancolor=yellow!10,badukpansize=7][L]
\KSBadukContinue
\BlackFirstN{C14,F16,D10,F3,C6,D2,H3,C3,E3,E2,F4,G4,G3,F2,J4}[1]  % [1] 생략가능
\end{ksbadukpan}
\end{minipage}
\hfill
\begin{minipage}{.45\textwidth}
\centering
\caption{때아닌 곤마}
\begin{ksbadukpan}[badukpancolor=yellow!10,badukpansize=7][DL]
\KSBadukContinue
\WhiteFirstN{F5,E4,G5,J6}
\SaveKSBaduk{jamna}
\end{ksbadukpan}
\end{minipage}
\end{figure}

\begin{figure}[!h]
\centering
\caption{응징을 못하면 응징당한다}
\begin{ksbadukpan}
\LoadKSBaduk{jamna}
\WhiteFirstN{G8,C4,G10}
\end{ksbadukpan}
\end{figure}

\end{document}