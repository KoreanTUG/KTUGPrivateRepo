%% part of kotexguide
\let\textmj\textrm
\newcommand{\xample}[1]{\bgroup\small #1\egroup}
\newcommand\pgffamily{%
  \ifEUCmode
    \hfontfamily{pg}\selectfont
  \else
    \SetAdhocFonts{utpg}{utgt}%
  \fi
}

\chapter{\kotex/euc의 의의와 사용법}

\section{\kotex/euc의 의의}

\subsection{문자의 범위}

\kotex/euc는 EUC-KR 한글만을 처리한다. EUC-KR 한글이라고 통칭하는 이 
한글 인코딩 방식은 특히 KS X 1001의 완성형 한글과 한자 및 기호문자를
지칭하는 것으로 사용하겠다. Windows 운영체제의 기본 한글 인코딩은
CP949라고 불리는 것으로 이 EUC-KR 완성형 한글을 확장한 ``확장완성형''
인코딩이다.

EUC-KR은 한글 2350자만을 정의하고 있으므로 유니코드 현대 한글 음절문자
11172자의 일부밖에는 표현하지 못한다. 우리의 언어 문자생활에 필요한 거의
대부분의 문자가 들어 있다고 하나 한글의 특성 중 하나인 자유로운 자모
결합에 의한 표현력을 희생하지 않을 수 없으며, (비록 표준 정서법에는
어긋나더라도) 특별한 목적을 위해서 표현하고자 하는 일부 글자를 찍을
수 없다는 한계를 갖는다.

이러함에도 불구하고, EUC-KR의 필요성도 많다. 가장 많은 사람이
사용하는 윈도우즈 운영체제에서 아무런 불편없이 입출력할 수 있다는 점,
그리고 무엇보다 기존에 이 완성형 한글로 작성된 문서와 문헌이 워낙
방대하다는 점이다.

\kotex 이 EUC-KR 지원을 하는 이유는 궁극적으로 이 때문이다. 기존에
작성된 문서들에 대한 호환성을 유지하는 것. 또한, 장차 \kotex은
EUC-KR 문서를 안전하게 유니코드로 포팅하는 방법을 발전시켜\footnote{%
  아직까지는 이 호환이 완전히 제공되지 않는다.}
장기적으로 \kotex/utf로의 이행을 권장하기 위해서이다. 

완성형 한글의 범위는 부록을 참고하라. 

\subsection{\kotex/euc의 한계}

\kotex/utf와는 달리 \kotex/euc가 호환을 위해서 제공되기 때문에
\kotex/utf의 많은 기능들을 생략하거나 지원하지 않는다. 그 때문에
이 두 형식의 문서가 출력 결과에 있어서 동일하지 않다. 

\kotex/utf의 몇 가지 옵션, 예를 들면 \texttt{nonfrench, finemath}
등은 지원하지 않는다. \kotex 의 자랑 중 하나인 미세 간격 기능은
euc 버전에서는 동작하지 않는다. 

\subsection{하위 호환성}

종래의 한글 문서들이 \texttt{hangul} 또는 \texttt{hfont}
패키지를 이용하고 있음이 엄연하고, 따라서 이 패키지들에 대하여 하위호환성을
제공한다. 그러나 되도록 이 문서에서 설명하는 바와 같이 \texttt{kotex}
기본 패키지를 이용하는 방식으로 이행해가는 것이 바람직하지 않겠는가
기대한다. \verb|\usepackage{hfont}|는 \verb|\usepackage[euc]{kotex}|으로,
\verb|\usepackage{hangul}|은 \verb|\usepackage[euc,hangul]{kotex}|으로
바꾸어 쓰면 거의 동일하게 동작할 것이다.

한글 라텍의 특징 중 하나였던 글꼴 선택 명령 체계는 여전히 지원된다.
그러나 유니코드와의 호환성을 위하여 이 부분에 대한 설명은 최소한에
그치고 호환성 높게 문서를 작성하는 방법을 위주로 설명하고자 한다.

기본 글꼴은 더이상 UHC 폰트를 사용하지 않게 되었다. 그러나 사용자
수준에서 글꼴 처리에 대해서는 걱정하지 않아도 좋다. 

종래 \texttt{hangul} 패키지와 함께 제공되던 \texttt{khyper} 패키지는
제외되었다. \kotex/euc가 \texttt{hyperref}을 잘 지원하며,
심지어 필수적으로 요구하기까지 하는 까닭이다. 이 부분이 아마
가장 많이 달라진 부분 중 하나일 것이다. 

\subsection{사용설명서에 관하여}

이 사용설명서의 \kotex/euc 부분은 한글 라텍의 ``한글 라텍 길잡이''
문서를 거의 그대로 인용한 것이다. 

\section{\kotex/euc의 사용 설명}
\label{sec:use}\index{코텍@ko.TeX/euc!사용법}

이 단원에서는 라텍 자체의 기본적인 사용법에 대한 설명을 피하고,
한글 라텍에 관계되는 부분만을 간단히 다루려 한다. 

\subsection{한글 문서의 틀}
\index{한글 라텍!문서의 틀}

문서는

\verb|\documentclass[추가선택]{문서종류}|

\noindent
로 시작한다. \texttt{추가선택}에는 라텍의
\texttt{문서종류}(class)에 종속적인 추가 선택을 지정한다.

문서 종류 지정 후에 한글 라텍 꾸러미 사용을 지정한다.

\verb|\usepackage[euc,추가선택]{kotex}|

\noindent%
한글 라텍의 \texttt{추가선택}의 종류는 다음과
같다.\index{한글 라텍!추가 선택}
\begin{list}{\textbullet}{\parsep0pt\itemsep0pt\topsep0pt\partopsep0pt}
\item \texttt{euc}: EUC-KR 한글을 처리하는 \kotex/euc에서는 
이 옵션이 가장 처음에 있어야 한다.
\item \texttt{hanja}: 문서의 단원명이나 날짜가 한자로 표기된다.
\item \texttt{hardbold}: \MF{}에 의해 만들어지는 두꺼운 글씨체를
  쓰도록 한다.
\item \texttt{nojosa}: 자동 조사 처리 기능을 억제하여 텍의 기억
  용량 사용을 감소시킨다.
%\item 판형: 책의 크기를 지정하는 선택 사항입니다.  책의 크기는 라텍
%  자체의 추가 선택으로 지정되는데, 이 추가 선택에는 한국의 출판
%  업계에서 사용되는 판형에 해당하는 크기가 없으므로 한글 라텍의 추가
%  선택으로 지정해야 합니다.  한국에서 사용되는 책의 판형에는 표
%  \ref{tab:papersize}\과 같은 종류가 있습니다.\index{판형} 국전지의 크기는
%  939$\;\times\;$636mm이고 46전지의 크기는
%  788$\;\times\;$1091mm입니다.

%  \begin{table}[htbp]
%    \centering
%    \begin{tabular}{|l|r@{$\;\times\;$}l*{4}{|l}|}\hline
%      판형명칭 & \multicolumn{2}{c|}{크기[mm]} & 사용종이 & 쪽수/전지 & \\\hline\hline
%      \ttfamily 국판 & 148 & 210 & 국전지 & 32 & 교과서, 단행본 \\\hline
%      \ttfamily 국배판 & 210 & 297 & 국전지 & 16 & 잡지 \\\hline
%      \ttfamily 국반판 & 105 & 148 & 국전지 & 64 & 문고 \\\hline
%      \ttfamily 타블로이드판 & 257 & 364 & 4x6전지 & 16 & 정보신문 \\\hline
%      \ttfamily 사륙판 & 128 & 182 & 4x6전지 & 64 & 문고 \\\hline
%      \ttfamily 사륙배판 & 182 & 257 & 4x6전지 & 32 & 참고서 \\\hline
%      \ttfamily 사륙반판 & 94 & 128 & 4x6전지 & 128 & \\\hline
%      \ttfamily 사륙배배판 & 254 & 374 & 4x6전지 & & \\\hline
%      \ttfamily 신국판 & 152 & 225 & 국전지 & 32 & 단행본 \\\hline
%      \ttfamily 대국전판 & 172 & 244 & 국전지 & & \\\hline
%      \ttfamily 크라운판 & 176 & 248 & 4x6전지 & 36 & 사진집 \\\hline
%      \ttfamily 삼십절판 & 125 & 205 & 4x6전지 & 60 & 단행본 \\\hline
%      \ttfamily 신서판,삼륙판 & 103 & 182 & 4x6전지 & 80 & 문고 \\\hline
%      \ttfamily 삼오판 & 84 & 148 & 4x6전지 & & \\\hline
%    \end{tabular}
%    \caption{판형의 종류}
%    \label{tab:papersize}
%  \end{table}
\end{list}

\texttt{kotex} 이외의 다른 꾸러미를 함께 쓰고자 할 경우에는
반점(\texttt{,})으로 구분하여 이름을 쓰거나 따로 \verb|\usepackage|
명령을 쓴다.  \texttt{kotex}의 추가 선택은 문서 종류의 추가
선택에서 지정할 수도 있고 여기서 지정할 수도 있다. 추가 선택의
처리 방식은 다음과 같다.\index{추가 선택 처리 방식}
\begin{itemize}
\item \verb|\documentclass|의 추가 선택은 전역적(global)이다.  어느
  문서 종류나 꾸러미에서든지 이 추가 선택을 처리할 수 있다.
\item \verb|\usepackage|의 추가 선택은 울안적(local)이다. 그
  꾸러미에서 이 추가 선택을 처리할 수 있어야 한다.
\end{itemize}

다음의 예에서 보는 것처럼

\verb|\usepackage[euc]{kotex,a4}|

\noindent
에서는 지정된 추가 선택 \texttt{euc}를 두 꾸러미 \texttt{kotex}과
\texttt{a4}가 모두 처리할 수 있어야 한다.  \texttt{a4} 꾸러미는
\texttt{euc}이라는 추가 선택을 처리할 수 없으므로 오류가 발생하게
된다.  다음과 같이 각기 따로 지정하거나

\verb|\usepackage[euc]{kotex}|

\verb|\usepackage{a4}|

\noindent
아니면 문제가 되는 추가 선택 \texttt{euc}를 \verb|\documentclass|에서
전역적 추가 선택으로 지정하면 오류가 발생하지 않는다.

\verb|\begin{document}|

\noindent
로 문서를 시작하여 쓰고자 하는 글을 작성한 뒤

\verb|\end{document}|

\noindent
로 문서 작성을 마친다.  (그림 \ref{fig:docstructure}\을 참조)

\begin{figure}[htb]
  \centering
  \begin{tabular}{|p{6cm}cp{5.5cm}|}\hline
    \verb|\documentclass[추가선택]{문서종류}| & &
    \\[10pt]
    \verb|\usepackage[euc,추가선택]{kotex}| & $\leftarrow$ &
    {\pgffamily 한글 문서 틀잡기를 위해 \texttt{[hangul]}
    추가선택을 추가할 수 있다.}
    \\
    \verb|\usepackage[추가선택]{꾸러미}| & $\leftarrow$ &
    {\pgffamily 필요에 따라 사용될 각종
    \texttt{꾸러미}를 지정한다.}
    \\
    $\ldots$ & $\leftarrow$ & {\pgffamily
    꾸러미뿐만 아니라 문자를 출력하지 않는 다른 명령들을 여기에서
    선언할 수 있다.}
    \\
    \verb|\begin{document}| & $\leftarrow$ &
    {\pgffamily 이전까지를 전문
    영역(pre\-amble)이라고 한다.}
    \\
    \ttfamily 나라의 말이 중국과 달라서 한자로는 서로 통하지
    아니하므로 이런 까닭으로 어리석은 백성들이 말하고자 하는 바가
    있어도 마침내 제뜻을 능히 펴지 못하는 사람이 많으니라.  내가
    이를 불쌍히 여겨 새로 스물여덟자를 만드나니 &
    $\leftarrow$ & {\pgffamily 문서의 작성은
    \textrm{\TeX}의 관습에 따른다. 즉, 중첩된 공간 문자는 하나로
    처리된다. 줄바꿈 문자가 한번 나오면 공간 문자로 처리되나, 두번
    이상 중첩되면 하나의 줄바꿈 문자 역할을 한다. 등등.} \\
    \verb|\end{document}| & $\leftarrow$ &
    {\pgffamily 이 명령이 나온 후의 모든 문장은
    무시된다.} \\\hline
  \end{tabular}
  \caption{한글 라텍 문서의 기본 구조}
  \label{fig:docstructure}
\end{figure}

이제 \texttt{latex}이나 \texttt{pdflatex}으로 파일을 돌리면 된다.


\subsection{문서의 모양새 설정}
\label{sec:layout}\index{문서의 모양새}

전반적인 문서의 종류를 결정하기 위해 다음 문장을 단 한 번 맨 처음에 쓴다.
\begin{verbatim}
\documentclass[추가선택]{문서종류}
\end{verbatim}
\texttt{문서종류(class)}에는 \texttt{article} 이외에 \texttt{report},
\texttt{book}, \texttt{letter}, \texttt{slides} 등이 있다. 각
\texttt{문서종류}는 \texttt{추가선택}에 의해 모양새가 달라질 수
있다. 즉, 종이의 규격을 설정하는 \texttt{a4paper},
\texttt{letterpaper}, $\cdots$ 및 글자의 크기를 정해주는
\texttt{10pt}, \texttt{11pt}, \texttt{12pt} 등등이 있다.  일반 영문
라텍의 문서 종류 설정은
\cite{Kopka:LE91,Lamport:1985:LDP,Mittelbach:2004:LC}에 잘 설명되어
있다.

위와 같이 라텍에서 제공하는 기본적인 문서 종류 이외에 추가로 꾸러미(package)를
사용하고자 할 경우에는 \verb|\usepackage| 명령을 쓴다.
\begin{verbatim}
\usepackage[추가선택]{꾸러미}
\end{verbatim}
우리말의 처리와
같이, 라텍에서 제공되지 않는 기능을 쓸 경우에 \texttt{kotex.sty}을
쓰는 것과 같다. 각 꾸러미는 자체의 \texttt{추가선택}이 있다.
각종 꾸러미의 사용 방법 및 \texttt{추가선택}의 종류는
\cite{Mittelbach:2004:LC}에 자세히 설명되어 있다.

한글 라텍으로 우리말을 처리하게 하려면 다음과 같이 한다.
\begin{verbatim}
\usepackage[euc]{kotex}
\end{verbatim}
\texttt{kotex.sty}를 \texttt{euc}이외의 추가선택 없이 쓰면 (종래 \texttt{hfont.sty}를
쓴 것과 같이) 문서 틀잡기가
우리말에 어울리는 한글화를 전혀 반영하지 않고 단순히 한글 및 한자, 상징
기호를 포함한 문서에서 우리말 처리가 제대로 될 수 있게만
한다. 자동조사는 동작하지만, 자동조사의 동작을 위해서는 \pkg{hyperref}\를
필요로 한다. 만약 순수하게 한글을 식자할 의도로 자동조사를 필요로
하지 않는다면 \texttt{nojosa} 옵션을 주면 된다.\footnote{%
  그러므로 엄밀히 말하면 예전의 \texttt{hfont}와 동일한
  것은 \texttt{[nojosa]\{kotex\}}이다.} 

영문 문서와 달리 우리말 문서에서는 줄간격을 특별히 취급해야 한다.
이는 우리말 글자들이 영문자와 비교해서 모양이 복잡함으로 인해 발생하는
문제로서, 라텍의 애초값으로 우리말 문서를 만들 경우 각각의
\pagename\이 글자로 밀집된 인상을 받는다.  이는 글을 읽는 독자의
시각을 분산시키는 악영향을 미치므로 다음과 같은 명령으로 줄간격을
조절하는 것이 좋다.\index{줄간격}
\begin{verbatim}
\renewcommand{\baselinestretch}{1.3}
\normalsize
\end{verbatim}
또는
\begin{verbatim}
\linespread{1.3}
\end{verbatim}
이 명령은 줄간격을 지정하는 레지스터인 \verb|\baselineskip|을 1.3배로
증가시키는 작용을 하며 \verb|\normalsize|와 같이 글자체의 크기 바꿈
명령이 나타날 때 이러한 변경이 효력을 발휘한다.  \kotex/euc에
\texttt{[hangul]} 추가선택을 더하여 한글 문서 틀잡기를 활성화하면
\verb|\baselinestretch|의 기본값이 1.333으로 자동 설정된다.  이
설정은 문서의 어느 곳에서든지 사용자의 취향에 따라 변경될 수 있다.
줄간격을 좀더 쉽게 제어할 수 있도록 하는 \pkg{hsetspace}\가
제공된다. 이것은 \pkg{setspace}\를 불러들여서 문서 내의 각주나
떠다니는 개체 내의 줄간격을 재설정하는 등 몇 가지 추가적인 기능을
제공한다. \pageref{sec:setspace} 페이지 \ref{sec:setspace}절을
참고하라.


\subsection{글자체 선택}
\label{sec:selectfont}\index{한글 라텍!글자체 선택}

문서를 작성할 때, 어떤 특정한 부분의 글자체를 현재 쓰고 있는 글자체와
달리 하고 싶을 때가 있다.  글자의 두께를 진하게 한다든지, 크기를
크게 한다든지, 다른 모양을 쓴다든지, 또는 설계(design)가 전혀 다른
글자체를 쓴다든지 등등.

이러한 글자체 바꿈을 위해 라텍은 여러가지 글자체 바꿈 명령을
제공한다.  라텍 209에서 제공됐던 \verb|\rm|, \verb|\sf|, \verb|\sl|,
\verb|\it| 등등의 글자체 바꿈 명령들은 모두가 같은 차원에서 글자체를
바꾸기 때문에 여러가지 단점을 가지고 있었다. 즉, 글자체의 설계가
\verb|\rm|이 아닌 \verb|\sf|를 계속 써 오다가 그 글자체의 기울인 모양인
\verb|\sl|로 글자체를 바꾸면 바뀐 글자체가 \verb|\sf|의 \verb|\sl|이
아니라 \verb|\rm|의 \verb|\sl|이 된다.  이는 이런 글자체 바꿈 명령이
일단 글자체를 애초값으로 복귀시킨 후 글자체를 바꾸기 때문이다.
이러한 단점을 보완하기 위해 1989년부터 Frank Mittelbach와 Rainer
Sch\"opf는 ``새로운 글자체 선택 방식''\textsf{(NFSS: New Font
  Selection Scheme)}을 고안하기 시작하여 현재 라텍 버전 2$\epsilon$에
채택되어 있는 \NFSS{}를 만들었다.


\subsubsection{영문 글자체 선택 방법}
\index{라텍!글자체 선택}

\NFSS{}는 글자체를 다섯가지 요소로 구분하여 이 요소들을 각각
독립적으로 선택할 수 있게 한다.\index{NFSS2} 그 다섯가지 요소는
다음과 같다.\index{글자체!구성 요소}

\begin{description}
\item[글자체 부호화(encoding)] 다른 알파벳을 쓸 때 각 나라별로
  \texttt{codepage}가 정해져 있다. 라텍에서는 다른 프로그램와
  마찬가지로 각 글자와 0--255 사이의 글자체 위치를 연관시킴으로써 이를
  해결한다.  라텍에서는 이를 ``부호화 방법''이라 한다.
\item[글자체 가족(family)] 글자체 가족은 같은 설계 원칙을 갖는다.
  가족 내에서는 크기(size)와 무게(weight), 폭(width), 모양(shape)으로
  구성원이 구별된다.
\item[글자체 모양(shape)] \ifEUCmode\else\ungremph\fi 
  글자체 가족의 구성원을 구별하는 중요한
  속성이다. ``\textup{곧은(upright) 모양}''과 ``\textit{굽은(italic)
    모양}'' ``\textsl{기운(slanted) 모양}'' 혹은 대문자의 크기를
  70\%정도로 축소하여 소문자 대신 쓰는 ``\textsc{작은키(Small Caps)
    모양}''등으로 구별된다. 글자체 모양에 있어서 그 중요도는
  떨어지지만 ``외곽형({\fontshape{ol}\selectfont outlined}) 모양''과
  ``그림자(shaded) 모양'' 및 ``곧은 이탤릭({\fontshape{ui}\selectfont
    upright italic}) 모양'' 등도 글자체 모양에 속한다.
    \ifEUCmode\else\regremph\fi
\item[글자체 무게(weight) 및 폭(width)] 무게와 폭은 서로의 조합으로
  하나의 글자체 속성을 형성한다. 무게는 글자의 두께로서 결정되며
  ``보통 두께(medium)'', ``가벼운 두께(light)'', ``무거운 두께(bold)''
  $\ldots$ 등으로 나뉜다. ``가벼운 두께''에도 ``아주 가벼운
  두께(extra light)'', ``매우 가벼운 두께(ultra light)'', ``보통
  가벼운 두께(light)'', ``약간 가벼운 두께(semi light)'' 등등으로
  분리되며 ``두꺼운 두께''도 이런 식으로 세분된다. 폭은 ``보통의
  폭(medium width)'', ``확장된 폭(extended width)'' 등이 있다.
\item[글자체 크기(size)] 글자체 크기는 인쇄기의 점을 단위로
  표현된다. 인쇄기의 점을 텍에서는 \texttt{pt}라는 단위로 나타내며
  1인치는 72.27\texttt{pt}이다.  라텍에서는 글자체 크기가 1.2의
  승으로 구별된다. 이렇게 정한 배경에는 후에 축소 복사를 할 때의
  편이함이 있다. 즉 A5 크기의 책자를 만들 경우, A4 종이에
  $1.44=\sqrt2$의 글자체 크기를 선택하여 인쇄한 후 A5 크기로
  축소한다.
\end{description}

글자체 선택에는 명령형과 선언형이 사용될 수
있다.\index{명령형}\index{선언형} 명령형은 \verb|\textbf{...}|와
같이 한가지 변수를 요구하며, 선언형은 \verb|\bfseries|와 같이 변수를
요구하지 않는다. 선언형은 다음과 같이 환경 형태로도 쓰일 수
있다.\index{환경형}

\begin{center}
\begin{tabular}{|m{55mm}cm{5cm}|}\hline
  \begin{verbatim}몇몇 단어들은 \begin{bfseries}
두껍게 \end{bfseries}
인쇄된다.\end{verbatim}
  & $\Rightarrow$ &
  몇몇 단어들은 \begin{bfseries} 두껍게 \end{bfseries}
  인쇄된다. \\\hline
\end{tabular}
\end{center}

문서의 애초 글자체는 명령형 \verb|\textnormal|이나 선언형
\verb|\normalfont|로 선택되며, 이 명령/선언은 문서의 글자체를
초기화시킴으로써 항상 같은 모양을 선택할 수 있게 한다.

문서의 글자체 가족에는 로만(roman)체, 산세리프(\textsf{sans serif})체,
타자(\texttt{type\-writer})체가 있다.  이들은 각각 다음의 명령형과
\begin{verbatim}
\textrm \textsf \texttt
\end{verbatim}
선언형으로
\begin{verbatim}
\rmfamily \sffamily \ttfamily
\end{verbatim}
선택된다.  로만 글자체는 일반적으로 문서의 기본 글자체로 선언되어
있다.

문서의 글자체 시리즈(series)는 폭과 두께의 두 요소로 구성되며, 다음과
같은 명령 및 선언형으로 선택된다.
\begin{verbatim}
\textmd   \mdseries
\textbf   \bfseries
\end{verbatim}
애초값은 \verb|\textbf|가 확장된 두꺼운 두께(bold extended)이고
\verb|\textmd|는 보통의 폭과 보통의 두께이다.

글자체 모양은 애초값인 곧은 모양을 비롯하여 굽은 모양, 작은키 모양,
기울인 모양이 있는데 이들은 각각 다음의 명령/선언형으로 선택된다.
\begin{verbatim}
\textup   \upshape
\textit   \itshape
\textsc   \scshape
\textsl   \slshape
\end{verbatim}
이 중에 굽은 모양과 곧은 모양 사이의 전환은 \verb|\emph| 명령 및
\verb|\em| 선언으로 이루어진다.


글자체 크기 선택을 위해서는 10가지의 선언형이 사용된다. (명령형은
없다.)

\noindent
\begin{tabular}{ll}
\verb|\tiny|         & \tiny 미소한 크기       \\
\verb|\scriptsize|   & \scriptsize 각본철 크기 \\
\verb|\footnotesize| & \footnotesize 주석 크기 \\
\verb|\small|        & \small 작은 크기        \\
\verb|\normalsize|   & \normalsize 보통 크기   \\
\verb|\large|        & \large 큰 크기          \\
\verb|\Large|        & \Large 좀더 큰 크기     \\
\verb|\LARGE|        & \LARGE 아주 큰 크기     \\
\verb|\huge|         & \huge 거대한 크기       \\
\verb|\Huge|         & \Huge 아주 거대한 크기  \\
\end{tabular}

\bigskip

위와 같은 외형 글자체 바꿈 명령들의 행동 방식은 내형 글자체 바꿈
명령에 의해 달라진다.\index{외형 글자체 바꿈 명령}%
\index{내형 글자체 바꿈 명령} 예를들어
\begin{verbatim}
\renewcommand{\familydefault}{cmss}
\end{verbatim}
와 같은 내형 글자체 바꿈 명령을 쓰면 글자체 가족의 애초값이
\texttt{cmss}(Computer Modern Sans Serif)로 바뀜으로써 주석 및 머릿말
등, 외형 글자체 바꿈 명령만으로는 바뀌지 않는 부분까지의 글자체 가족이
영향을 받는다.

라텍에서 애초값은 표~\ref{tab:defaultfont}\와 같이 설정되어 있다.

\begin{table}
\centering
\caption{라텍에서 글자체 애초값}\label{tab:defaultfont}
\begin{tabular}{>{\ttfamily}l@{\quad}>{\ttfamily}l}\hline
\textmj{내형 글자체} & \textmj{애초값} \\\hline
\verb|\encodingdefault| & OT1 \\[3pt]
\verb|\familydefault| & \verb|\rmdefault| \\
\verb|\seriesdefault| & m \\
\verb|\shapedefault| & n \\[3pt]
\verb|\rmdefault| & cmr \\
\verb|\sfdefault| & cmss \\
\verb|\ttdefault| & cmtt \\[3pt]
\verb|\bfdefault| & bx \\
\verb|\mddefault| & m \\[3pt]
\verb|\itdefault| & it \\
\verb|\sldefault| & sl \\
\verb|\scdefault| & sc \\
\verb|\updefault| & n \\\hline
\end{tabular}
\end{table}


\subsubsection{우리말 글자체 선택 방법의 개요}

\medskip \HFSS{}\footnote{Hangul Font Selection Scheme}%
는 \NFSS{}를 바탕으로 두고 있다. 우리말 부호화의 특성에
따라 한국어 부호화(encoding) 방식을 지정하는 \verb|\hfontencoding|과 한국어
글자체 가족(font family)을 지정하는 \verb|\hfontfamily|가 추가 되었으며 글자체 선택
방식도 우리말 글자체에 맞도록 변경되었다.\index{HFSS}
\begin{description}
\item[글자체 부호화] 우리말의 부호화는 \texttt{H}로 설정되어 있고
  \NFSS{}에서 사용되는 글자체 부호화 설정 모듬 명령
  \verb|\fontencoding|을 우리말화한 \verb|\hfontencoding|을 쓴다.
  우리말 글자체의 부호화는 사용자의 입장에서 볼 때 달리 설정할 이유가
  없고, 영어 부호화 방식과의 구별에 그 주 목적이 있으며 \HFSS{}의
  작동시 우리말 글자체의 선택을 우리말에 맞게 하도록 하는 기능을
  한다.
\item[글자체 가족] \NFSS{}에서와 같이 우리말의 글자체 가족은 글자체의
  설계 원칙에 따라 구분이 되고, 우리말의 특성에 의해 두 요소로
  표현된다.

  첫번째 요소는, \verb|\kscfamily|로서 우리말의 상징 문자와 한글 및
  한자를 구별한다. 이 모듬 명령도 사용자의 입장에서는 쓸 일이 거의
  없고, 주로 한글 라텍의 내부 처리 과정에서 사용된다.

  두번째 요소는 우리말의 글자체 가족을 나타내는
  \verb|\hfontfamily|로서 우리말의 글자체 가족을 바꿀 때 사용자가
  직면하게 되는, 실질상 우리말의 글자체 가족을 대표하는 요소이다.
\item[글자체 무게 및 폭] 작동 방식과 내용이 \NFSS{}와 동일하다.
  추가된 사항은 글자체 시리즈의 접두사 \texttt{S}로서 이는 우리말의
  \verb|\bfseries|가 \texttt{softbold}로 처리되도록 한다.
\item[글자체 모양] \NFSS{}와 같다.
\item[글자체 크기] \NFSS{}와 같다.
\end{description}


\subsubsection{우리말 글자체의 가족 선택}

\medskip KS X 1001의 부호 체계에 의한 상징 문자/한글/한자의 가족
구분은 글자의 부호값으로 정해지므로 특별한 명령/선언 모듬 명령을
필요로 하지 않는다.  단지 미래에 추가될 수도 있는 가능성을 위해 혹은
운영 체계 관리자를 위해, 설치된 한글 라텍의 특성에 따라 다음의 모듬
명령으로 상징 문자/한글/한자를 구분 설정할 수 있도록 하였다.
\begin{verbatim}
  \kscfamily{상징문자}{한글}{한자}
\end{verbatim}
이는 각각
\begin{verbatim}
\symboldefault
\hanguldefault
\hanjadefault
\end{verbatim}
로 설정되어 있으며 그 값은 각각 \texttt{s}, \texttt{w}, \texttt{h}이다.

우리말 문서에서의 실질적인 『우리말 글자체 가족』은%
\index{우리말 글자체 가족}
\begin{verbatim}
  \textmj  \mjfamily
  \textgt  \gtfamily
  \texttz  \tzfamily
\end{verbatim}
로 선택되며 이들은 각각
\begin{verbatim}
  \textrm  \rmfamily
  \textsf  \sffamily
  \texttt  \ttfamily
\end{verbatim}
와 같은 작용을 한다. (왼쪽은 명령형, 오른쪽은 선언형)

위와 같은 모듬 명령들은 우리말 가족을 바꾸어 주는 모듬 명령
\verb|\hfontfamily|를 사용한다.
\begin{verbatim}
  \hfontfamily{우리말가족}
\end{verbatim}
\texttt{우리말가족}에는 다음과 같은 글자체가 있다.
\index{한글 라텍!글자체 종류}

\begin{center}
\begin{tabular}{|>{\ttfamily}ll>{\ttfamily}ll>{\ttfamily}ll|}\hline
  \multicolumn{6}{|l|}{기본가족:}\\
  mj  & (명조)     & gt  & (고딕)    & tz  & (타자)      \\\hline
  \multicolumn{6}{|l|}{추가가족:}\\
  gr  & (그래픽)   & gs  & (궁서)     & pg  & (필기)     \\
  yt  & (옛글)     & bm  & (봄글씨)   & pn  & (펜글씨)   \\
  ph  & (펜흘림)   & sh  & (신문)     & vd  & (바다)     \\
  jmj & (자모명조) & jgt & (자모고딕) & jnv & (자모노벨) \\
  jsr & (자모소라) &     &            &     &            \\ \hline
%  dn  & (디나루)   & pga & (필기a)    &     &            \\\hline
\end{tabular}
\end{center}

위와 같은 우리말 가족들은 상징 문자/한글/한자의 세 가족을 대표하는
외형 가족으로서, 문서에서 \texttt{우리말가족}을 \texttt{tz}으로
지정하였을 경우 한글 라텍은 한글/상징 문자/한자의 순으로 각각
\begin{verbatim}
 wtt, wtt, wtt
\end{verbatim}
의 내형 가족을 쓰도록 되어 있다.  외형 가족과 내형 가족의 연관은
사용자가 다음의 명령으로 문서의 어디에서든지 변경할 수
있다.\index{MapHangulFamily@\verb/\MapHangulFamily/}%
\index{외형 가족}
\begin{verbatim}
 \MapHangulFamily{외형가족}{한글가족,상징문자가족,한자가족}
\end{verbatim}
\texttt{tz}의 경우 \texttt{내형가족}과의 연관은 다음과 같은 명령으로
이루어진다.
\begin{verbatim}
 \MapHangulFamily{tz}{wtt,wtt,wtt}
\end{verbatim}

\subsubsection{우리말 글자체 시리즈}

문서의 글자체 시리즈의 선택에는 \NFSS{}가 그대로 적용되며 라텍을 쓸
때는 한글 라텍 버전 0.96까지 사용되었던 \texttt{softbold}를 위해 글자체
시리즈의 접두사 \texttt{S}가 추가되어 보통 두께의
글자체를(\texttt{mdseries}) 세 번 중첩 인쇄하는 기능을
한다.\index{softbold@\texttt{softbold}} 그러므로 라텍 문서에서 다음의 명령은
\begin{verbatim}
  \renewcommand{\bfdefault}{Sbx}
혹은
  \fontseries{Sbx}
\end{verbatim}
우리말의 글자체 두께를 \texttt{softbold}로 처리하도록 하며 영문
글자체의 두께는 \texttt{bx}가 되도록 한다. \texttt{kotex.sty}의 추가
선택 \texttt{softbold}는 위와 같은 기능을 애초값으로 설정한다.

\subsubsection{우리말 글자체의 모양과 크기}

우리말 글자체의 모양/크기는 \NFSS{}와 같다.

\subsubsection{우리말 글자체의 애초값}

문서의 애초 글자체는 \NFSS{}에서와 같이 다음의 명령/선언형으로 선택된다.
\begin{verbatim}
 \textnormal   \normalfont
\end{verbatim}
한글 라텍에서 애초값은 다음과 같이 설정되어 있다.
\begin{verbatim}
                     한글 라텍
  \hencodingdefault  H
  \encodingdefault         
  \symboldefault     s
  \hanguldefault     w
  \hanjadefault      h
  \mjdefault         mj    
  \gtdefault         gt    
  \tzdefault         tt    
  \hfamilydefault    \mjdefault
\end{verbatim}
위와 같은 애초값들은 운영 체계 관리자가 \texttt{hfont.cfg}에서
변경 설정할 수 있고, 사용자도 문서의 서두에서 재정의할 수 있다.

\subsubsection{그외 우리말 글자체의 모듬 명령}
\label{sec:hangulmacro}

한글 라텍 버전 0.96까지 써 오던 두 글자로 구성되는 글자체 바꿈 명령은
기본 글자체 이외에는 모두 삭제되었다.  이들은 라텍 버전 209 식의
글자체 바꿈 명령에 해당하며 사용을 회피하는 것이 좋다.
\begin{verbatim}
  \mj
  \gt
  \tz
\end{verbatim}

그 반면에 우리말로 구성되는 글자체 바꿈 명령은 글자체의 가족을
선택하는 선언형으로서 계속 존재한다.%
\index{한글 라텍!글자체 바꿈 명령}

\begin{verbatim}
                    한글 라텍  
  \명조     \textmj \mjfamily
  \고딕     \textgt \gtfamily
  \타자     \texttz \tzfamily
  \그래픽           \hfontfamily{gr}
  \궁서             \hfontfamily{gs}
  \신문             \hfontfamily{sh}
  \필기             \hfontfamily{pg}
  \펜글씨           \hfontfamily{pn}
  \펜흘림           \hfontfamily{ph}
  \봄글씨           \hfontfamily{bm}
  \옛글             \hfontfamily{yt}
  \자모명조         \hfontfamily{jmj}
  \자모고딕         \hfontfamily{jgt}
  \자모노벨         \hfontfamily{jnv}
  \자모소라         \hfontfamily{jsr}
\end{verbatim}
\texttt{\textbackslash hfontfamily}나 \texttt{\textbackslash
  fontfamily}로 글자체를 지정한 후에는 \texttt{\textbackslash
  selectfont}로 글자체를 선택해야 글자체가 바뀐다.

\medskip
\verb|\usefont|는 \NFSS{}의 모듬 명령으로서 글자체의 부호화와 가족,
시리즈 및 모양을 한번에 정할 수 있도록 한다.
\begin{verbatim}
  \usefont{부호화}{가족}{시리즈}{모양}
\end{verbatim}
이 모듬 명령에서 첫 번째 변수인 글자체 부호화가 \verb|\hfontencoding|과
같을 경우에는 두 번째 변수를 \verb|\hfontfamily|로 취급한다.
그러므로 다음의 선언은
\begin{verbatim}
  \usefont{H}{gs}{m}{n}
\end{verbatim}
우리말 글자체 중 궁서체의 보통 두께, 보통 모양을 쓰도록 한다. 이때
영문의 경우도 두께와 모양에서 우리말을 따르게 된다.

다음과 같은 명령은 \NFSS{}에 대응하는, 우리말을 위한 선언으로서
\HFSS{}의 작동에 필수불가결한 모듬 명령이다.  이들은
\verb|\begin{document}| 전에서만 선언할 수 있는 성질의 것인 만큼,
  사용자의 입장에서는 보통의 경우 신경쓰지 않아도 된다.
\begin{verbatim}
\DeclareErrorHFont{부호화}{가족}{두께}{모양}{크기}
\DeclareHFontSubstitution{부호화}{가족}{두께}{모양}{크기}
\end{verbatim}

\subsubsection{\kotex/utf와 \kotex/euc의 글자체 선택 방식}

\kotex/utf는 HFSS를 채택하지 않고 있다. 따라서 \verb|\hfontfamily|나
\verb|\hfontseries| 등에 상당하는 명령이 없다. 그 대신 한글 폰트의
글꼴 가족을 지정하되, 다른 폰트 속성 등은 NFSS를 따르도록 하고 있다.

또한 글꼴 가족에 대한 개념도 달라서, euc버전이 한글/심벌/한자를 묶어서
하나의 가족으로 부르는 반면, utf에서는 한글 글꼴 가족과 한자 및 기타 문자
글꼴 가족으로 분리시켜 두고 이들을 대응하는 방식을 선택하였다. 

예를 들어, 본문의 rm/sf/tt 글꼴 세트를 바꾸기 위해서 euc 버전에서는
\begin{verbatim}
\MapHangulFamily{mj}{XXbt,XXbt,XXbt}
\MapHangulFamily{gt}{XXgt,XXgt,XXgt}
\MapHangulFamily{tz}{XXtz,XXtz,XXtz}
\mjfamily, \gtfamily, \tzfamily
\end{verbatim}
와 같이 지정해야 할 것이나, 이것을 utf 버전에서는
\begin{verbatim}
\SetHangulFonts{XXbt}{XXgt}{XXtz}
\SetHanjaFonts{XXbt}{XXgt}{XXtz}
\end{verbatim}
과 같이 할당한 다음, \verb|\rmfamily|, \verb|\sffamily|, \verb|\ttfamily|로
쓰도록 하고 있다. euc 버전과 utf 버전 사이의 호환성에
가장 큰 걸림돌이 이 폰트 선택 명령이다. 한글 라텍 방식대로 
한다면 듣도보도못한 \verb|\XXfamily|가 나오지 말라는 법이
없는데, 이것을 일일이 모두 지원할 수 없는 까닭이다. 

실제로, HFSS로 인하여, \verb|\XXfamily| 명령을 남발하게 되면
이것은 심각한 호환성의 문제를 불러일으킨다. 다른 곳에서는 컴파일이
불가능한 상황이 생기는 것이다. 우리는 rm, gt, tz라는
세 가지 한글 글꼴 패밀리도 너무 많다고 생각한다. 그러므로, 이 세 종류의
글꼴 가족 이외에는 반드시 \verb|\hfontfamily{XX}\selectfont|를
일관되게 쓰는 것이 좋으며, 본문 글꼴을 교체하려 할 때는 \verb|\MapHangulFamily|를
사용하도록 할 것을 권장한다. 또한 \verb|\textmj|와 같은 명령은
아예 일체 사용하지 않는 것이 좋다고 생각한다. 
euc 버전에서도 \verb|\rmfamily|, \verb|\sffamily| 등
NFSS 명령만을 사용해서 문서를 작성하는 것이 좋다. 

\subsection{우리말 숫자 모듬 명령}
\label{sec:nummacro}\index{한글 라텍!숫자 모듬 명령}

한글 라텍에서는 우리말 숫자를 숫자 환경에서 쓸 수 있도록 우리말 숫자
모듬 명령을 제공한다. 우리말 숫자 모듬 명령에는 다음과 같은 것들이
있다.

\begin{verbatim}
\jaso(ㄱㄴㄷㄹ)  \gana(가나다라)
\ojaso(㉠㉡㉢㉣) \ogana(㉮㉯㉰㉱)
\pjaso(㈀㈁㈂㈃) \pgana(㈎㈏㈐㈑)
\onum(①②③④)  \pnum(⑴⑵⑶⑷)
\oeng(ⓐⓑⓒⓓ)  \peng(⒜⒝⒞⒟)
\hnum(하나둘셋)  \Hnum(첫째둘째)
\end{verbatim}

예를들어, 다음과 같이 정의를 하면
\begin{verbatim}
\renewcommand{\labelenumi}{\gana{enumi})}
\renewcommand{\labelenumii}{\jaso{enumii}.}
\renewcommand{\labelitemi}{☞}
\end{verbatim}
\begin{tabular}[c]{|m{45mm}cm{4cm}|}
\multicolumn{1}{l}{다음의 원전(text)은} & &
\multicolumn{1}{l}{이렇게 나타납니다.} \\\hline
\begin{verbatim}
\begin{enumerate}
\item 한글
  \begin{enumerate}
  \item 현대어
  \item 고어
  \end{enumerate}
\item 漢字
  \begin{enumerate}
  \item 韓國語
  \item 中國語
  \item 日本語
  \end{enumerate}
\item 상징 기호
  \begin{itemize}
  \item ㉿
  \item ₩
  \item ㅫ
  \end{itemize}
\end{enumerate}
\end{verbatim}
& $\Rightarrow$ &
\renewcommand{\labelenumi}{\gana{enumi})}
\renewcommand{\labelenumii}{\jaso{enumii}.}
\renewcommand{\labelitemi}{☞}
\begin{enumerate}
\item 한글
  \begin{enumerate}
  \item 현대어
  \item 고어
  \end{enumerate}
\item 漢字
  \begin{enumerate}
  \item 韓國語
  \item 中國語
  \item 日本語
  \end{enumerate}
\item 상징 기호
  \begin{itemize}
  \item ㉿
  \item ₩
  \item ㅫ
  \end{itemize}
\end{enumerate} \\\hline
\end{tabular}

\bigskip

나열환경의 항목 머리에 대하여 \kotex/utf는 좀더 편리한 방식을
지원한다. 그렇지만 \kotex\-/euc에 대해서 별도의 추가 지원 패키지를 마련하지
않았다. 즉 \kotex/euc에서는 \pkg{enumerate}의 환경 옵션으로 한글
항목머리를 지시할 수 없다.\utf\footnote{%
  \texttt{enumitem} 2.0 이후 버전은 항목 머리 옵션을
  추가하는 것이 매우 쉬우므로 \kotex/euc에서도 설정해서 쓸 수
  있을 것으로 생각한다.}

\subsection{우리말 각주 판짜기}
\label{sec:footnote}
\index{각주 판짜기}

영문 라텍에서의 기본적인 각주 판짜기 방식을 사용하면 각주 번호가
표시되는 각주 첫 줄이 들여 쓰여진다.  각주 번호에는 수학식에서 멱수를
나타내는 윗첨자가 사용되고 각주 번호와 각주문을 구분하는 공간이
없다.  이와 같은 판짜기 방식 이외에도 출판된 우리말 문서를 보면
다른 방식의 판짜기를 볼 수 있다.

\cite{LeeJW91}\은 각주면을 본문의 왼쪽에 맞추는 방식과 각주면을
본문의 안쪽으로 들여 넣는 방식으로 각주면 판짜기를 대분하고 각주문의
첫줄을 들여쓰기와 내어쓰기에 의해 다시 세분하는 방식으로 각주 판짜기를
분류한다.

한글 라텍의 각주 판짜기는 영문에서의 방식을 그대로 사용한다.  그러나
사용자의 선호도에 따른 각주 판짜기 방식을 쉽게 적용할 수 있도록
\texttt{hangulfn.sty} 패키지를 따로 제공한다. \pageref{sec:fn}페이지를
보라.

\subsection{조사 이형태, 매개 모음, 지정사 탈락}
\label{sec:autojosa}
\index{자동 음운 변화}
\index{조사 이형태}
\index{매개 모음}
\index{지정사 탈락}

\verb|\ref|나 \verb|\pageref| 등의 교차 참조와 \texttt{\textbackslash
  cite} 등의 문헌 인용을 사용하면 숫자나 이름표가 자동으로 생성된다.
이 경우 문서를 작성하고 있는 순간에는 이 숫자나 문자가 무엇인 지를 알
수 없다. 그러므로 그 다음에 오게 되는 주격 조사가 `\texttt{가}'가
될 지 아니면 `\texttt{이}'가 될 지를 결정할 수 없다.  사용되는
변이형은 자동으로 생성되는 숫자의 음가나 이름표의 마지막 음절이 무슨
받침을 갖고 있는가에 따라 다르다.  매개 모음 ``으''의 삽입이나 지정사
``이''의 탈락도 같은 경우다.  \kotex/euc는 다음과 같은
음운 이형태를 자동 처리할 수 있도록 했다.\index{자동 음운 변화}
\begin{verbatim}
    은/는     이/가    을/를    와/과
    으로/로    으로서/로서      으로써/로써
    이라/라
\end{verbatim}
이를 처리해 주는 모듬 명령은 각각 다음과 같다.
\begin{verbatim}
    \은 \는         \이 \가         \을 \를         \와 \과
    \으로 \로       \으로서 \로서   \으로써 \로써     \이라 \라
\end{verbatim}
이 가운데 `\verb|\라|, \verb|\이라|'는 \kotex/utf와의
호환을 위하여 마련된 명령이며, 아래 설명하는 지정사 탈락을
부분적으로 지원하는 것이다.

매개 모음 ``으''의 삽입은 명령 ``\texttt{\{\textbackslash ㅡ\}}''를
통해 구현다.  명령 이름은 ``으''의 모음 ``ㅡ''이다.  예를 들어
``\texttt{\textbackslash 로}''의 정의는 ``\texttt{\{\textbackslash
  ㅡ\}로}''이다.  매개 모음의 삽입은 동사의 굴절에서도 나타나지만
이를 자동으로 처리해야 할 경우는 없을 것으로 보인다.  그 이유는
동사가 굴절을 하지만 어간이 없는 동사가 없기 때문에 매개 모음의 앞에는
정해진 동사 어간이 있기 때문이다.  한글 라텍에서 제공되지 않은 경우의
매개 모음의 처리는 ``\texttt{\{\textbackslash ㅡ\}}''를 삽입하여 해결할
수 있다.

지정사 ``이''의 탈락은\footnote{지정사 ``이다''는 학자에 따라 조사로
  취급하기도 하고 상태 동사의 하나로 분류하기도 한다.  여기서
  국어학적 측면은 논외로 한다.}  ``교수(이)시니 선생님이시니?''  혹은
``라텍이냐 워드(이)냐가 관건이다'' 등과 같이 많은 경우가 있다.
경우의 수가 많은 이유는 지정사가 동사의 한 하위 부류로서 다양한 굴절을
보이기 때문이다.  따라서 탈락 가능한 지정사에는 다음의 일반적인
명령을 사용한다.  명령의 이름에 사용된 문자는 ``이''의 모음
``ㅣ''이다.
\begin{verbatim}
    {\ㅣ}
\end{verbatim}

%예:
%\begin{tabular}{|l@{$\;\;\;\rightarrow\;\;\;$}l|}\hline
%  \ttfamily
%  참고한 문헌은 \textbackslash cite\{Choi99\}\textbackslash ㅣ다. &
%  참고한 문헌은 \cite{Choi99}\ㅣ다.\\\hline
%\end{tabular}
%\medskip

\utf \kotex/utf에서는 `\verb|\ㅣ|'와 `\verb|\ㅡ|'를 채택하지 않고
있다. 그러므로 호환가능한 문서를 작성할 때 이 부분에 주의를 요한다. 그
대신 ``지정사 탈락''에 대비하기 위하여 `이라(고/서/\ldots)' 단 한 경우만을
지원하며, 이것은 `\verb|\라|' `\verb|\이라|'로 구현하고 있다.

\verb|\cite| 명령으로 문헌 인용을 할 때 \texttt{plain.bst} 모양새
파일을 쓰면 인용의 이름표가 숫자로 표현되고 \texttt{halpha.bst} 모양새
파일을 쓰면 인용의 이름표가 문자열로
표현된다.\footnote{\texttt{alpha.bst}를 쓰면 우리말이
  깨진다.}\index{자동 조사!인용} \index{halpha.bst} 후자의 경우,
한글 라텍은 음운 변화 처리 기능을 쓰는데 있어서 문자열에 다음과 같은
제한을 두고 있다.\index{자동 음운 변화 처리}
\begin{enumerate}
\item 문자열은 두 바이트 이상이어야 한다.  (\texttt{halpha.bst}를
  쓰면 문자열은 보통 다섯 글자 이상이 된다. 우리말은 한 글자가 두
  바이트 혹은 세 바이트로 구성되므로 우리말 한 글자는 문제의 대상에서
  제외된다.)

  \begin{tabular}[c]{|lcp{5cm}|}\hline
    옳음: \verb|\bibitem[HKD]{홍길동}...| & & \\
    틀림: \verb|\bibitem[H]{홍길동}...| & $\leftarrow$ &
    {이름표 \texttt{H}는 한 바이트이므로 착오를 발생시킨다.}
    \\\hline
  \end{tabular}

  본문에서 \verb|\cite{홍길동}\을|이라 하면 ``\xample{[HKD]를}''로
  찍힌다.
\item 문자열의 마지막 음절은 한글과 한자, 우리말 자소, 숫자 그리고
  영문자로 구성된다.  즉, 자소를 제외한 상징 기호는 자동 음운 변화
  처리에서 고려의 대상이 되지 않는다.\footnote{상징 기호가 인용의
    이름표로 나올 일이 없을 것으로 생각한다.} 다시 말하자면, 자동
  음운 변화 처리 기능은 작동하나 그 결과는 문법에 맞지 않을 수
  있다.  이형태의 결정은 실질 형태소의 마지막 음절이 `ㄹ'이 아닌
  종성인 경우로 간주된다.
\item 문자열의 마지막 글자가 영문자일 경우에, 자동 음운 변화 처리
  기능은 낱말의 음가를 고려하지 않고 마지막 문자의 우리말 음가로
  이형태를 결정한다.\footnote{모든 영어 낱말의 우리말 음가를 올바로
    분석하기는 매우 힘들다.}
\end{enumerate}

\begin{figure}
\centering
  \begin{tabular}[c]{|lcp{5.5cm}|}\hline
    옳음: \verb|\bibitem[set]{홍길동}...| & $\leftarrow$ &
    {영문자 이름표 \texttt{set}는 마지막 문자 \texttt{t}에 의해
      이형태를 결정한다. \texttt{t}의 우리말 음가는 중성으로
      끝나는 음절이다. \texttt{set}의 우리말 음가는 \texttt{세트}로서 역시
      마지막 음절이 중성으로 끝난다.  문헌 인용의 결과는
      \texttt{[set]로}이다.} \\
    틀림: \verb|\bibitem[ping]{홍길동}...| & $\leftarrow$ &
    {영문자 이름표 \texttt{ping}은 마지막 문자 \texttt{g}가
      이형태를 결정된다. \texttt{g}의 우리말 음가는 중성으로
      끝나는 음절이다. 그러나 \texttt{ping}의 우리말 음가는 \texttt{핑}으로서
      종성으로 끝나는 음절이다.  문헌 인용의 결과는
      \texttt{[ping]로}이다.} \\\hline
  \end{tabular}
\end{figure}

\texttt{halpha.bst}를 사용할 경우에 인용의 부호는 저자의 이름 중 앞 세
글자와 책이 출판된 연도의 뒷 두 숫자로 구성되거나 (저자가 한 명일 때),
첫 번째 등장하는, 최대한 세 명의 저자 이름 중 첫 글자와
\verb+\etalchar+ 그리고 책이 출판된 연도의 뒷 두 숫자로 구성되므로
(저자가 여러명일 경우) 세 번째의 제한은 무시될 수 있다.  보통의
경우에도 \verb|\cite|의 이름표(label)는 일련 번호이거나
(\texttt{plain.bst}) \texttt{저자이름+책출판년도}를 약식으로
표현한다 (\texttt{halpha.bst}).  그러므로 보통은 발생하기 힘든
경우에 해당한다.  그러나 참고 문헌 작성 시에는 이런 사실을 알고
있으므로써, 발생할 가능성이 있는 오류를 사전에 방지하는 것이 좋다.

색인 작성 시에 \texttt{makeidx.sty}을 쓰면 \verb|\see| 명령이
정의된다.  이 명령을 쓸 때는 \verb|\see|의 변수가 빈 칸을 포함하게
되면 자동 음운 변화 처리 기능에 의해 오류가 발생한다.  빈칸 대신에
``\texttt{\textasciitilde}''나
``\texttt{\textbackslash\textvisiblespace}''를 쓰면 문제가 발생하지
않는다.\index{자동 음운 변화!색인} 예: 그림~\ref{fig:autojosait}

\begin{figure}
\centering
\begin{tabular}[c]{|lp{5.5cm}cp{5cm}|}\hline
  옳음: & \verb!\index{절|see{첫장~첫절}}!
  \verb!\index{절|see{첫장\ 첫절}}! &
  $\leftarrow$ & `\texttt{\textasciitilde}'나
  `\texttt{\textvisiblespace}'는 모두 빈 칸을
  의미한다. `\texttt{\textasciitilde}'는 앞뒤의 단어를 같은 줄에
  출력하도록 하고 `\texttt{\textvisiblespace}'는 다른 기능없이 빈 칸만
  출력한다. \\
  틀림: & \verb!\index{절|see{첫장 첫절}}! & & \\\hline
\end{tabular}

\texttt{출력:} 첫장 첫절\textit{\seename}
\caption{색인 작성}\label{fig:autojosait}
\end{figure}


\verb|\pagename|, \verb|\chaptername|과 같이 우리말화 한 한글 이름들도
자동 음운 변화 처리를 할 수 있다.  그러므로
`\verb|\pagename\가|'라고 쓰면 ``\xample{\pagename\가}''라고
출력된다.

\subsection{우리말 이름}
\label{sec:names}\index{우리말 이름}

한글 라텍에서 사용되는 각종 이름들을 일률적으로
\verb+\ksnamedef+ 모듬 명령에 의해 변경될 수 있도록
하였다.\index{ksnamedef@\verb+\ksnamedef+} 
이러한 ``한글 이름''을 사용하려면 \texttt{[hangul]} 옵션을
지정하여야 한다.

예를 들어, 라텍에서
사용되는 `\refname{}'의 이름 `Reference'는 \verb+\refname+에 지정되어
있다.  이를 우리말화 하는 방법은 다음과 같다.
\begin{verbatim}
  \ksnamedef{refname}{참고~문헌}
\end{verbatim}

\utf \kotex/utf 에서도 \texttt{[hangul]} 옵션을 준 경우라면
이 명령을 사용할 수 있다.

표~\ref{tab:names}에는 한글 라텍에서 한글화한 ``라텍 이름의 한글 및
한자화''가 나열되어 있다.  이 한글화는
\texttt{\textbackslash{}begin\{document\}} 앞에서 위에 예를 든 것처럼
\texttt{\textbackslash{}ksnamedef}로 재정의하면 원하는대로 변경된다.

\begin{table}[htbp]
  \leavevmode
  \centering
  \begin{tabular}{|r|c|c|c|}\hline
               라텍명령   & 한글 라텍(한글)  & 한글 라텍(漢字)  \\\hline\hline
           \texttt{today} & 1994년 3월 6일  & 1994年 3月 6日  \\\hline
        \texttt{enclname} & 동봉물          & 同封物          \\\hline
          \texttt{ccname} & 사본            & 寫本            \\\hline
      \texttt{headtoname} & 받는이          & 受信人          \\\hline
         \texttt{seename} & \verb+\+을 참고 & \verb+\+을 參考 \\\hline
    \texttt{contentsname} & 차\~{}례        & 目\~{}次        \\\hline
\texttt{listfigurename} & 그림\~{}차례    & 그림\~{}目次    \\\hline
 \texttt{listtablename} & 표\~{}차례      & 表\~{}目次      \\\hline
         \texttt{refname} & 참고\~{}문헌    & 參考\~{}文獻    \\\hline
       \texttt{indexname} & 찾아보기        & 索\~{}引        \\\hline
       \texttt{tablename} & 표              & 表              \\\hline
    \texttt{abstractname} & 요\~{}약        & 要\~{}約        \\\hline
        \texttt{bibname}  & 참고\~{}문헌    & 參考\~{}文獻    \\\hline
    \texttt{appendixname} & 부록            & 附錄            \\\hline
           \texttt{ksTHE} & 제              & 第              \\\hline
        \texttt{partname} & 편              & 篇              \\\hline
     \texttt{chaptername} & 장              & 章              \\\hline
     \texttt{sectionname} & 절              & 節              \\\hline
      \texttt{colorlayer} & 환등판\~{}색깔  & 幻燈版\~{}色相  \\\hline
    \texttt{glossaryname} & 용어집         & 語\~{}彙        \\\hline
        \texttt{pagename} & 페이지          & 쪽              \\\hline
      \texttt{figurename} & 그림            & 그림            \\\hline
  \end{tabular}
  \caption{라텍 이름의 한글 및 한자화}
  \label{tab:names}
\end{table}

\texttt{\textbackslash{}ksTHE}는 라텍에 정의되어 있지 않고,
한글 라텍에서 단원의 숫자 판짜기에 사용되도록 도입된 이름 모듬
명령이다.  이는 라텍 이름을 한글/한자화 함으로써 발생하는 문제점들을
해결하기 위해 사용된다.  예를 들어 \texttt{\textbackslash{}part}의
일련 번호는 단원 숫자의 한글화로 인해 ``편 I''과 같이 나오며
한글 라텍에서는 \texttt{\textbackslash{}ksTHE}를 도입하여 ``제 I 편''과
같은 식으로 짜여지도록 하고 있다.

한글 라텍에서는 \texttt{\textbackslash{}part}와
\texttt{\textbackslash{}chapter}, \texttt{\textbackslash{}appendix},
\texttt{\textbackslash{}section}까지를 위와 같이 한글화하고 있으며,
각각 다음과 같이 설정되어 있다.\index{한글 라텍!단원 판짜기}

\medskip
\begin{tabular}{|r|rcl|}\hline
         단원 종류 & 숫자 앞 & 숫자 & 숫자 뒤 \\\hline\hline
      \verb+\part+ & \verb+\ksTHE~+ & \verb+\thepart+    &
                                                \verb+~\partname+  \\\hline
   \verb+\chapter+ & \verb+\ksTHE~+ & \verb+\thechapter+ &
                                             \verb+~\chaptername+  \\\hline
                   &                & \verb+\thesection+ & article \\
  \raisebox{1.5ex}[-1.5ex]{\texttt{\textbackslash appendix}} &
             \raisebox{1.5ex}[-1.5ex]{\texttt{\textbackslash
                               appendixname\textasciitilde}} &
                                  \verb+\thechapter+ & book/report \\\hline
   \verb+\section+ & \verb+\ksTHE~+ & \verb+\thesection+ &
                                             \verb+~\sectionname+  \\\hline
\end{tabular}
\medskip

이를 사용자의 편의에 따라 재설정하기 위해서는 다음의 모듬 명령을
사용한다.
\begin{verbatim}
  \kscntformat{단원이름}{앞}{뒤}
\end{verbatim}
예를 들어 \texttt{\textbackslash chapter}를 ``첫째 마당''과 같은 식으로
짜기 위해서는 전문(preamble)---\texttt{\textbackslash begin\{document\}} 앞---에서 다음과
같이 지정한다.
\begin{verbatim}
  \renewcommand{\thechapter}{\Hnum{chapter}}
  \ksnamedef{chaptername}{마당}
  \kscntformat{chapter}{}{~\chaptername}
\end{verbatim}


\subsection{그외의 우리말화}
\label{sec:hangulization}

우리말의 분철은\index{한글 라텍!분철} 붙임표가 사용되지 않으므로 영어권
언어와는 다른 특별한 처리 방법이 요구된다.
\def\hidewidth{\clearocplists\hskip\hideskip}

영어 문서에서 단원 제목에 이은 첫 문단에서는 들여쓰기를 하지 않는다.
그러나 한글 문서에서는 모든 문단에 들여쓰기가 적용되는 것으로
보인다.  (이는 한글 문서 작성의 표준이 규정된 것으로 보이지 않으나
한국의 출판 서적을 볼 때 그렇게 판단된다.)  첫 문단 들여쓰기를 하는
국가는 한국 뿐만은 아닌 것으로 보이며, 텍 운영 체계에서
\texttt{indentfirst} 꾸러미를 사용하여 이를 해결하도록 되어 있다.
즉, 문서의 전문 영역에서
\begin{verbatim}
    \usepackage{indentfirst}
\end{verbatim}
를 지정하면 첫 문단 들여쓰기가 쉽게 이루어진다.

또한 들여쓰기의 정도에 있어서도 영어 문서와 우리말 문서는 차이가 있는
것으로 보인다.  (역시 정해진 표준이 있는 것으로 보이지 않고
출판사에서 관습적으로 들여쓰기의 양을 정해 쓰는 것으로 보인다.)
들여쓰기의 양은 한글 글자 두개의 넓이 즉 전각 2배각을
쓰는 것이 적당해 보인다. 이는 많은 한국어 서적에서 보이는 전각 들여쓰기보다
큰 값이지만 \cite{LeeJW91}\과 KS A 0104에서 채택되고 있는 값이고
우리말 문서에 적당하게 보인다.  이 들여쓰기의 양은 다음과 같이
설정할 수 있다.
\begin{verbatim}
    \parindent=2em
\end{verbatim}

\section{한글화}
\label{sec:hangul}

EUC-KR 부호화 문자를 처리하는 한글 라텍에서 발생하는 문제점들은 대부분이
문자의 범주에 기인한다.  텍의 명령은 더 이상 \texttt{expand} 될 수
없을 때까지 \texttt{expand} 되는 반면에 문자는 더 이상 \texttt{expand}
되지 않는 범주에 속한다.  영어 알파벳 문자는 모두 문자 범주에
속하므로 예를 들어 ``word''를 구성하는 네 개의 문자는 \texttt{expand}
후에도 그대로 남는다.  반면에 EUC-KR로 부호화된 ``낱말''은 네 개의
부호가 두 개씩 쌍을 이루고 있으며 각 쌍이 두 개의 한글 음절 ``낱''과
``말''을 형성한다.  EUC-KR 부호화 방식은 한 음절을 구성하는 두 개의
바이트 값에 의해 한국어 음절 값이 정해지므로 한글 라텍에서는 첫 번째
바이트를 두 번째 바이트를 인수로 취하는 명령으로 인식하도록 하고
있다.  그 결과 한글 음절은 \texttt{expand} 된 후에 소위 말하는 원시
명령으로 구성되는 일련의 명령 집합체로 변한다.

문자 범주를 기대하는 명령의 변수가 한글일 때 라텍은 오류를
유발한다.  특히 \texttt{\textbackslash label}이나 \texttt{\textbackslash
  cite} 명령 등의 변수가 한글일 때가 그러하다.  한글 라텍에서는 표준
문서 양식에서 사용되는 이와 같은 명령들을 수정하여서 한글 라벨을 사용할
수 있도록 했다.  그러나 다른 패키지에서 이런 명령들을 재정의하면
또다시 문제가 발생할 수 있다.

문제가 되는 것 중에 \pkg{hyperref}\가 있다. pdf 제작에 있어서는
거의 필수적인 이 패키지가 한글 라벨 이름을 받아들이지 않는다.
pdf를 제작하거나 \kotex/utf와의 호환을 고려한다면 되도록 한글
라벨 이름을 쓰지 않는 것이 좋다. \utf

\pkg{showkeys}\는 \texttt{\textbackslash label},
\texttt{\textbackslash ref}, \texttt{\textbackslash pageref},
\texttt{\textbackslash cite}, \texttt{\textbackslash bibitem} 명령을
수정하여 내부적으로 사용되는 표지를 출력하는 기능을 수행한다.
한글 라텍에서는 이 패키지를 한글화하여 \texttt{showhkeys.sty}의
이름으로 제공한다.

참고 문헌의 인용에 사용되는 표지를
\texttt{showkeys} 패키지와는 다른 방식으로 출력하는
\texttt{showtags} 패키지를 한글화한 파일은 \texttt{showhtags.sty}이다.

%\texttt{hyperref} 패키지는 주로 \texttt{pdflatex}과 함께 쓰여 PDF 문서
%내에서의 교차 참조 사이를 항해할 수 있는 기능을 제공합니다.
%한글 라텍에서는 이 패키지도 한글화 하여 \texttt{khyper.sty}로
%제공합니다.  다만 bookmarks 기능은 제대로 구현하지 못했습니다.  그
%이유는 한글 문자의 부호화가 EUC-KR이기 때문이며 PDF에서는 유니코드를
%요구합니다.  bookmarks의 한글화는 EUC-KR 부호화 문자를 출력하는
%수준에서 그쳤습니다.  따라서 \texttt{pdflatex}의 bookmarks 기능은
%사용하지 않는 편이 좋습니다.

%한글Λ에서는 UTF-8로 부호화된 한글이 16비트의 문자 범주로 처리됩니다.
%따라서 한글 라텍의 경우와는 달리 \texttt{expand}에 의한 오류가 발생하지
%않습니다.  그러므로 \texttt{showkeys}, \texttt{showtags},
%\texttt{hyperref}에서 요구하는 한글화 작업이 필요하지 않습니다.  만일
%한글Λ에서 위의 한글화된 패키지를 사용했을 때는 한글화 되지 않은 기존
%패키지만 사용되고 한글화 부분은 적용되지 않습니다.

