% arara: xelatex: { shell: yes }
\documentclass[a4paper,12pt]{article}

\usepackage[hmargin=1in,vmargin=1in]{geometry}
\usepackage{setspace}

\usepackage{amssymb}
\usepackage{kotex}

\usepackage[hcr,units]{kstextks}

\usepackage{hologo,url}

\linespread{1.3}

\usepackage{multicol}

\usepackage[most]{tcolorbox}
\tcbuselibrary{minted}

\newtcblisting{mycode}{%
	colback=red!5!white,
	colframe=red!75!black,
	listing engine=minted,
	minted language=latex,
	minted style=colorful,
	minted options={fontsize=\small,linenos,numbersep=3mm},
	colback=blue!5!white,colframe=blue!75!black,listing only,
	left=5mm,enhanced,
	overlay={\begin{tcbclipinterior}\fill[red!20!blue!20!white] (frame.south west)
            rectangle ([xshift=5mm]frame.north west);\end{tcbclipinterior}}
}

\newtcblisting{mycoderesult}{%
	colback=red!5!white,
	colframe=red!75!black,
	listing engine=minted,
	minted language=latex,
	minted style=colorful,
	minted options={fontsize=\small,linenos,numbersep=3mm},
	colback=blue!5!white,colframe=blue!75!black,%listing only,
	left=5mm,enhanced,
	overlay={\begin{tcbclipinterior}\fill[red!20!blue!20!white] (frame.south west)
            rectangle ([xshift=5mm]frame.north west);\end{tcbclipinterior}}
}


\begin{document}

\title{kstextks, KS X 1001의 기호문자}
\author{nanim}
\date{2015/08/05}

\maketitle 

\begin{abstract}
 KS X 1001:2004(cf. \url{https://ko.wikipedia.org/wiki/KS_X_1001})에서 정하는 기호문자와 특수문자의 매크로를 제공하고
 이 기호문자를 식자할 폰트를 지정할 수 있게 한다.
\end{abstract}


\section{사용법}

\begin{mycode}
 \usepackage{kstextks}
 \usepackage[hcr]{kstextks}
 \usepackage[symfont={<font>}]{kstextks}
 \usepackage[units]{kstextks}
\end{mycode}

\textbullet\ \texttt{hcr} 옵션은 모든 기호문자를 함초롬 바탕 LVT 폰트로 식자한다.
기호 문자 식자에 사용할 폰트를 지정하려면 \verb|symfont=폰트이름|으로 한다.

\textbullet\ \texttt{units} 옵션을 주면 일부 단위 문자(길이, 무게 등)도 매크로로 정의한다.
다음 절의 표에서 \texttt{\textbackslash textksdegree}(\textksdegree)부터는 이 옵션을 주어야 사용할 수 있는
명령이다.

\textbullet\ 이 스타일은 \hologo{XeLaTeX}이나 \hologo{LuaLaTeX}만을 지원하며 \hologo{pdfLaTeX}은 지원하지 않는다.


\section{기호문자 명령}

\parindent=0pt

\begin{multicols}{2}
\begin{singlespace}
 \DisplayAllSymbols
\end{singlespace}
\end{multicols}

\section{닮음기호}

\begin{mycoderesult}
 $\triangle \mathrm{ABC} \kssim \triangle \mathrm{A'B'C'}$
\end{mycoderesult}


\end{document}
