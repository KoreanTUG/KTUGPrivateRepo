% arara: xelatex
\documentclass[nanum,b5paper]{oblivoir}

\usepackage{fapapersize}
\usefapapersize{*,*,1in,*,1in,*}

\usepackage{ksmisc,kslinematters}
\usepackage{ifpxltex}
\IfpxlTeX* {p,x}
{
	\usepackage[normalem]{ulem}
}

\begin{document}

\title{ksmisc}
\author{Nova De Hi}
\maketitle

\section{kslinematters.sty}

space, carriage return, tab 문자의 catcode를 조작하여 ``보이는 대로'' 출력하도록 하는 명령과 환경을 제공한다.
\begin{boxedverbatim}
\obeythem, \disobeythem
\vobeyspaces, \vobeytabs, \vobeylines
\begin{ksobeys} ... \end{ksobeys}
\end{boxedverbatim}

다음은 예시이다.
\begin{verbatim}
\obeythem
  가 나다
  가

  가나다 	라
\disobeythem
\end{verbatim}
\begin{framed}
\obeythem
  가 나다
  가

  가나다 	라
\disobeythem
\end{framed}

\verb|\obeythem...\disobeythem|을 \verb|ksobeys| 환경으로 쓸 수 있다.
다만 이 명령/환경은 평문단에서 정상 동작한다.
% 예컨대 sectioning commands와 같이 vertical space를 스스로 조절하는 명령과는 오동작할 수 있다.

다음 패키지 옵션은 ksobeys의 동작을 제어한다.

\begin{description}
 \item [\texttt{noblankline}] 여러 개의 빈 줄을 하나의 빈 줄로 처리하게 한다.
 \item [\texttt{tabtosp}] tab 문자를 tabstop 위치까지 이동하는 것이 아니라 4개의 스페이스 문자로 치환한다.
 \item [\texttt{extcmd}] \verb|\section| 명령과 \verb|\subsection| 명령이 ksobeys 환경 안에서 동작하게 한다. 단 이 명령들은 오직 하나의 인자만을 가지는 것으로 간주한다.
\end{description}

\bigskip 

\begin{boxedverbatim}
\newlinecommand, \renewlinecommand
\end{boxedverbatim}

line의 끝까지를 인자로 받는 명령을 정의한다. 인자는 \verb|#1|.
\begin{verbatim}
\newlinecommand\testline{\uline{#1}}
이 명령은 현재 위치에서 \testline 줄 끝까지를 인자로 받는다.
여기서 줄이 바뀌었다.
\end{verbatim}
\newlinecommand\testline{\uline{#1}}
이 명령은 현재 위치에서 \testline 줄 끝까지를 인자로 받는다.
여기서 줄이 바뀌었다.

\section{preparefont.sty}

\begin{boxedverbatim}
\preparefontfile{Font_File_Name}{url}[local_saved_file_name]
\end{boxedverbatim}

\verb|--shell-escape|을 주어야 한다. 먼저 \verb|Font_File_Name|으로 폰트를 체크하고 만약 없으면 \verb|<url>|로부터 다운로드받는다. zip이거나 직접 폰트 파일 자체를 가리키는 url이어야 한다.
폰트의 존재 유무를 체크하기 위하여 \verb|luaotfload-tool|을 부르므로 이것이 실행가능해야 한다.

Windows에서는 KTUG 사설 저장소의 ktugbin 패키지 설치를 요구한다.
Mac이나 linux에서는 \verb|curl|과 \verb|unzip|이 필요하다.

첨부된 testpreparefont.tex와 게시물 ``preparefont.sty''\footnote{\url{http://doeun.blogspot.kr/2014/07/preparefontsty-dropbox.html}.}을 참고하라.

\section{ksmisc.sty}

\begin{boxedverbatim}
\ks_pop_left:NN <tlvar1> <tlvar2>
\ks_pop_right:NN <tlvar1> <tlvar2>
\ks_clist_set_from_tl:NN <clistvar> <tlvar>
\ks_seq_set_from_tl:NN <seqvar> <tlvar>
\end{boxedverbatim}

\verb|\ks_pop_left:NN|과 \verb|\ks_pop_right:NN|은 \verb|<tlvar1>|의 왼쪽(오른쪽) 끝의 item을
하나 꺼내어 \verb|<tlvar2>|에 넣고 \verb|<tlvar1>|에서 삭제한다.
Expl3에는 \verb|\seq_pop_left:NN|이 있지만 tl(token list)에 대해서는 \verb|\tl_head:N|와 \verb|\tl_tail:N|이 있는 대신 pop 명령이 없어서 만든 것이다. 이를 이용하여 tl을 간단한 stack처럼 활용할 수 있다.
그러나 이 명령은 \verb|\ks_seq_set_from_tl:NN|과 \verb|\ks_clist_set_from_tl:NN|을 작성하기 위한 것이기도 하다. clist와 seq의 막강한 데이터 조작 능력을 tl에 대하여 얻고자 하는 것이 이 명령의 목적이다.

다음 예를 참고하라. (\verb|\ks_for_loop:nnn| 명령은 ksforloop.sty를 요구한다. 이것은 \verb|\ksforloop|와 달리 루프 코드를 그루핑하지 않는 대신 nesting이 불가능하다.)
\begin{verbatim}
\ExplSyntaxOn
\NewDocumentCommand \test { m }
{
    \tl_set:Nn \l_tst { #1 }
    \ks_for_loop:nnn {}{\tl_count:N \l_tst }
    {
        \ks_pop_right:NN \l_tst \l_tmp
        \fbox{\l_tmp}
    }
}
\ExplSyntaxOff
\test{가나다라마바사아{자차}카타파하}
\end{verbatim}
\ExplSyntaxOn
\NewDocumentCommand \test { m }
{
	\tl_set:Nn \l_tst { #1 }
	\ks_for_loop:nnn {}{\tl_count:N \l_tst }
	{
		\ks_pop_right:NN \l_tst \l_tmp
		\fbox{\l_tmp}
	}
}
\ExplSyntaxOff
\test{가나다라마바사아{자차}카타파하}

\bigskip

\verb|\ks_clist_set_from_tl:NN|은 이름 그대로 token list의 각 item을 comma separated list로 만든다. \verb|\ks_seq_set_from_tl:NN|도 마찬가지로 tl의 item을 sequence로 만든다. \verb|<clistvar>|와 \verb|<seqvar>|는 전역적(global)으로 정의된다.
\begin{verbatim}
\ExplSyntaxOn
\NewDocumentCommand \testa { m }
{
    \tl_set:Nn \l_tst { #1 }
    \ks_clist_set_from_tl:NN \g_tst_cl \l_tst
    \clist_use:Nnnn \g_tst_cl {\과~} {\과~} {,~그리고~}
}
\ExplSyntaxOff
\testa{123456789{10}}
\end{verbatim}
\ExplSyntaxOn
\NewDocumentCommand \testa {m}
{
	\tl_set:Nn \l_tst {#1}
	\ks_clist_set_from_tl:NN \g_tst_cl \l_tst
	\clist_use:Nnnn \g_tst_cl {\과~} {\과~}{,~그리고~}
}
\ExplSyntaxOff
\testa{123456789{10}}

\end{document}