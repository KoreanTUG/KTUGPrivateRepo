%!TEX TS-program = arara
% arara: xelatex: { shell: yes }
\documentclass[polyglossia,a4paper,12pt,kosection]{oblivoir}

\usepackage{fapapersize}
\usefapapersize{*,*,1in,*,1in,*}

%\usepackage{polyglossia}
\setdefaultlanguage{korean}
\newfontfamily\hangulfont{NanumMyeongjo}[Ligatures=TeX]
\newfontfamily\hangulfontsf{NanumGothic}[Ligatures=TeX]
\newfontfamily\hangulfonttt{UnTaza}

\usepackage[refcmd]{ksjosaref}

\usepackage[most]{tcolorbox}
\tcbuselibrary{minted}
\newtcblisting{code}{%
	colback=red!5!white,
	colframe=red!75!black,
	listing engine=minted,
	minted language=latex,
	minted style=colorful,
	minted options={fontsize=\small,linenos,numbersep=3mm},
	colback=blue!5!white,colframe=blue!75!black,listing only,
	left=5mm,enhanced,
	overlay={\begin{tcbclipinterior}\fill[red!20!blue!20!white] (frame.south west)
            rectangle ([xshift=5mm]frame.north west);\end{tcbclipinterior}}
}

\newtcblisting{coderesult}{%
	colback=red!5!white,
	colframe=red!75!black,
	listing engine=minted,
	minted language=latex,
	minted style=colorful,
	minted options={fontsize=\small,linenos,numbersep=3mm},
	colback=blue!5!white,colframe=blue!75!black,%listing only,
	left=5mm,enhanced,
	overlay={\begin{tcbclipinterior}\fill[red!20!blue!20!white] (frame.south west)
            rectangle ([xshift=5mm]frame.north west);\end{tcbclipinterior}}
}


\usepackage{kotex-logo}

\def\pkg#1{\textsf{#1}}

\begin{document}

\pagestyle{hangul}

\title{\texttt{ksjosaref} 패키지}
\author{v0.2, nanim}
\maketitle

\begin{abstract}
\koTeX 을 사용하지 아니하는 상황에서 간이 자동조사가 동작하는 \verb|\ref|-류 명령을 정의한다.
\end{abstract}

\section{패키지}

\begin{code}
\usepackage[<options>]{ksjosaref}
\end{code}

현재 버전에서 사용가능한 옵션은 \verb|refcmd| 하나뿐이다. 이 옵션은 \verb|\ref|, \verb|\pageref|, \verb|\eqref| 명령을 이 패키지의 것으로 변경한다. 단, 이 기능은 실험적인 것이므로 일반적으로 사용하지 않는 것이 좋다.

\section{명령}

\begin{code}
\josaref[<type>][<delim>]{<ref>}<조사>
\end{code}

\subsection{type}
이 패키지가 지원하는 \verb|\ref|-류 명령은 다음 세 가지이다.
\begin{enumerate}\tightlist
\item \verb|\ref|
\item \verb|\pageref|
\item \verb|\eqref|
\end{enumerate}
이 중 \verb|\eqref|은 \pkg{amsmath} 패키지를 로드해야 한다. 이 각각을 첫 번째 옵션 인자로 지정한다.
\begin{enumerate}\tightlist
\item \verb|\josaref|
\item \verb|\josaref[page]|
\item \verb|\josaref[eq]|
\end{enumerate}

\subsection{delim}

\verb|\ref{fig:test1}|은 \josaref{fig:test1}와 같은 결과를 식자한다.
이 참조된 결과 숫자의 좌우에 적당한 delimiter를 주는 것이다.
\begin{description}\tightlist
\item [paren] 괄호를 친다. \josaref[][paren]{fig:test1}.
\item [bracket] 각진괄호를 친다. \josaref[][bracket]{fig:test1}.
\item [left=, right= ] 임의의 부호를 지정한다. \verb|\josaref[][left=\{,right=\}]{fig:test1}| \josaref[][left=\{,right=\}]{fig:test1}.
\end{description}

이것은 두 번째 옵션 인자이므로 반드시 첫 번째 옵션 인자 자리가 (비우더라도) 존재해야 한다.

\subsection{조사}

한글 조사를 쓴다. 은, 는, 이, 가, 을, 를, 와, 과, 로, 으로를 쓸 수 있다.

한편, 서술격조사 `이(다)'는 특별하게 취급하는데, 현재 버전에서 자동조사로 간주하는 것은 `(이)라', `(이)나', `(이)다'의 세 가지이다. 다른 것, 예를 들면 `(이)고'는 격식있는 문서에서는 `이'를 밝혀주는 관행이 아직 유지된다.

\medskip

\begin{coderesult}
\josaref{fig:test1}이나 \josaref{fig:test2}이나 \josaref{fig:test3}를 보아라.
이것은 \josaref{fig:test2}이다.
\end{coderesult}

주격조사 `이'와 서술격조사 어간 `이'는 그 뒤에 따르는 글자가 있느냐 없느냐로 판별한다.
혼동을 피하려면 주격조사를 일관되게 `가'로 써도 된다.

\medskip

\begin{coderesult}
\josaref{fig:test2}이, \josaref{fig:test2}이라는, \josaref{fig:test3}가
\end{coderesult}

\section{refcmd 옵션}

\verb|refcmd| 옵션을 주면 다음과 같이 된다.
\begin{itemize} \tightlist
\item \verb|\ref| 명령은 \verb|\josaref|와 동일하게 동작한다.
\item \verb|\pageref|와 \verb|\eqref|는 \verb+\josaref[page|eq]+ 옵션을 준 것과 동일하다.
\end{itemize}

다만, 이 (대용) 명령들은 delim 지정, 즉 두 번째 옵션 인자를 주는 기능은 동작하지 않는다.

\medskip

\begin{coderesult}
\ref{fig:test1}이, \pageref{fig:test2}가, \eqref{eq:3}
\end{coderesult}

다음 명령은 각각 원래의 \verb|\ref|-류 명령이 아닌 \verb|\josaref|의 대용 명령으로 사용할 것인지
그것을 취소할 것인지를 나타낸다.

\medskip 

\begin{code}
\josarefcmds, \nojosarefcmds
\end{code}

\section{변경 사항}

\hangfrom{\textbullet\ \ }2015/08/08. ver0.2. 조사명령 뒤에 아무 것도 없이 문단이 끝날 때 발생하던 에러를 없앴다.

%\section{알려진 문제점}

%\hangfrom{\textbullet\ \ }이 명령들은 조사가 당연히 있는 것으로 간주하기 때문에, 아예 아무런 조사도 없이 문단의 마지막에서 끝나는 경우에 문제를 일으킨다. 마침표를 찍어주자.

\clearpage

\section{테스트를 위한 figure}

%%% dummy figures

 \begin{figure}
 \centering
 \includegraphics[scale=0.3]{example-image-a}
  \caption{test}\label{fig:test1}
 \end{figure}

  \begin{figure}
 \centering
 \includegraphics[scale=0.3]{example-image-b}
  \caption{test}\label{fig:test2}
 \end{figure}

  \begin{figure}
 \centering
 \includegraphics[scale=0.3]{example-image-c}
  \caption{test}\label{fig:test3}
 \end{figure}

  \setcounter{figure}{28}

  \begin{figure}
 \centering
 \includegraphics[scale=0.3]{example-image-golden}
  \caption{test}\label{fig:test4}
 \end{figure}

\setcounter{equation}{2}
\begin{equation}
a^2 = b^2 + c^2 \label{eq:3}
\end{equation}

\end{document}
