% !TEX program = arara
% arara: xelatex: { shell: yes }
\documentclass[b5paper,nanum]{oblivoir}

\usepackage{fapapersize}
\usefapapersize{*,*,1in,*,1in,*}

\def\cs#1{\texttt{\textbackslash #1}}

\usepackage[all]{graphicsonthefly}
\usepackage{jiwonlipsum}
\input insbox

\begin{document}

\title{Graphics on the fly}
\author{Nova De Hi}
\date{v0.9.5 \today}

\maketitle

\begin{abstract}
웹상의 그림에 대해 url을 주면 다운로드받아서 문서에 넣자 하는 것입니다.
url로 주어진 그림을 가져와서 \texttt{graphicx} 패키지의 \cs{includegraphics} 명령으로
그림을 삽입하는 \cs{includegraphicsonthefly}와, url로 주어진 animated gif 그림을 가져와서
\texttt{animate} 패키지의 \cs{animategraphics} 명령으로 그림을 삽입하는 \cs{animatedgifonthefly} 두 개의 명령으로 이루어져 있습니다.
\end{abstract}

\tableofcontents*

\section{패키지 사용 선언}

\begin{boxedverbatim}
\usepackage[options]{graphicsonthefly}
\end{boxedverbatim}

줄 수 있는 옵션은 다음과 같습니다.
\begin{itemize}
\item \texttt{[renewall]}. 기본적으로 로컬 폴더에 그림이 있으면 인터넷에 연결하지 않는데 이 옵션이 있으면 항상 새로 그림을 받아오게 합니다.
\item \texttt{[animate]} 또는 \texttt{[all]}. \cs{animatedgifonthefly} 기능이 활성화됩니다. 기본값은 animate를 사용하지 않는 것입니다.
\item \texttt{[curl]}. 온라인 상의 그림을 가져오는 데 사용하는 유틸리티는 \texttt{wget}입니다. \texttt{curl}을 \texttt{wget} 대신 사용하도록 하려면 \texttt{[curl]} 옵션을 주면 되는데 완전한 호환성을 보증하지는 않습니다.
\item \texttt{[nocoalesce]}. animated gif를 변환할 때 기본적으로 \texttt{-coalesce} 옵션을 주게 되어 있습니다. 이 기능이 필요없을 때 지정합니다.
\end{itemize}

\section{요구 사항}

\begin{enumerate}\firmlist
\item    컴파일시 \verb|--shell-escape| 옵션을 주어야 합니다.
\item    Windows라면 \texttt{wget}을 별도로 설치해야 합니다. ktug 사설저장소에서 \texttt{ktugbin}을 설치하세요.
\begin{verbatim}
> tlmgr install ktugbin
\end{verbatim}

\item   \marginpar{[v0.9.5]} png, jpg, pdf, gif, bmp 그림을 처리합니다. gif, bmp 이미지는 ImageMagick이 설치되어 있어야 합니다. \cs{animatedgifonthefly} 명령을 사용하기 위해서도 ImageMagick \verb|magick|가 필요합니다. 맥의 경우 homebrew로 ImageMagick을 쉽게 설치할 수 있습니다. 윈도우즈는 imagemagick을 별도로 설치하세요. \url{http://imagemagick.org}. 일부 리눅스 시스템에서 imagemagick의 버전이 오래되어 \texttt{magick} 스크립트가 없고 \texttt{convert}만 있다면 이것을 사용하도록 패키지 옵션으로 \texttt{[convert]}를 지시해야 합니다.

\item    url은 그림의 확장명이 포함되어 있어야 합니다. 즉 단축 url과 같이 그림 확장명을 포함하지 않는 경우에는 정상 동작하지 않습니다.
\item    그림을 불러오려면 (로컬에 저장된 그림이 없을 때) 컴파일시 인터넷에 연결되어 있어야 합니다. 인터넷에 연결되지 않은 기계에서 동작하게 하는 방법을 찾지 못했습니다.
\end{enumerate}

\section{그림 불러오기}

\begin{boxedverbatim}
\includegraphicsonthefly[옵션]{그림이름}{url}
\end{boxedverbatim}

\begin{itemize}\firmlist
\item 옵션은 \cs{includegraphics}에 줄 옵션이고요
\item 그림이름은 다운받아서 로컬에 저장할 파일이름을 \dotemph{확장자 없이} 써넣습니다.
\item url은 url입니다.
\end{itemize}

이 명령은 \cs{includegraphics} 명령을 쓰고 마지막에 url 하나를 추가한 형태입니다.

이전에 이미 다운받은 파일이 있으면 그것을 사용합니다.
새로이 다운로드하도록 하려면
\begin{verbatim}
\includegraphicsonthefly*[width=\linewidth]{mygrf}{url}
\end{verbatim}
이렇게 별표붙은 명령을 주면 그 그림에 한하여 다시 다운받습니다.
한편, \texttt{[renewall]} 옵션을 패키지에 주면 문서 내의 모든 그림을 다 새로 다운로드받습니다.

\bigskip

\noindent\textbullet\ 예제:
\begin{verbatim}
\includegraphicsonthefly[width=\textwidth]{test}%
     {http://www.ktug.org/xe/files/attach/images/17577/ktugBanner-2014.png}
\end{verbatim}
\includegraphicsonthefly[width=\textwidth]{test}%
     {http://www.ktug.org/xe/files/attach/images/17577/ktugBanner-2014.png}


\section{Animated Gif}

이 기능은 패키지 옵션으로 \texttt{[animate]} 또는 \texttt{[all]} 지정하여야 사용할 수 있습니다.

animate 패키지를 이용하여 animated gif를 처리하는 시험판을 만들어 보았습니다. 역시 온라인 이미지를 가져와서 문서에 포함하는 것이 목적입니다.
\begin{boxedverbatim}
\animatedgifonthefly[옵션]{<fps>}{그림이름}%
                    {시작카운터}{마지막카운터}{url}
\end{boxedverbatim}
마지막에 url이 붙는 것을 제외하면 \cs{animategraphics} 명령과 쓰는 방법이 (거의) 똑같습니다. 그림 이름에 \verb|-|를 붙이지 않아도 되는 것을 제외하면요.

예를 들어 6프레임으로 이루어진 foo.gif의 url을 주면 다운로드받아서 모두 6장의 png로 변환합니다. 이것으로 애니메이트하는 것입니다. fps는 초당 프레임수입니다.

자세한 것은 animate 패키지 문서를 참고하시고요.

\begin{itemize}\firmlist
\item    처음 gif를 다운로드받을 때는 모두 몇 프레임인지 알 수 없습니다. 따라서 시작 카운터를 0으로 하고 마지막 카운터도 0으로 한 다음 컴파일을 시도합니다.
\item    한번 컴파일하고 나면 디렉토리에 모두 몇 개의 png파일이 생성되었는지 바로 알 수 있습니다. 마지막 숫자를 마지막 카운터 자리에 써넣습니다.
%\item    애니메이션은 Adobe Reader 이외의 pdf 뷰어에서는 효과를 볼 수 없는 것으로 압니다.
\item 굳이 animate할 필요 없이 첫 프레임만 그림으로 포함하려면 마지막카운터를 0 상태 그대로 두면 되겠습니다.
\end{itemize}

v0.6에서 마지막카운터에 별표를 쓸 수 있게 했습니다. 별표를 쓰면 프레임 수가 36장 이하일 때 알아서 체크하여 포함합니다. 36장을 넘는 gif는 36장까지만 포함합니다.

\bigskip

\noindent\textbullet\ 예제:
\begin{verbatim}
\animatedgifonthefly[width=4cm,autoplay,loop]%
        {10}%
        {butterfly}%
        {0}%
        {*}%
        {http://www.animatedgif.net/animals/bugs/butrfly_e0.gif}
\end{verbatim}
\animatedgifonthefly[width=4cm,autoplay,loop]%
        {10}%
        {butterfly}%
        {0}%
        {*}%
        {http://www.animatedgif.net/animals/bugs/butrfly_e0.gif}

\bigskip

로컬에 미리 다운로드받아둔 gif 파일이 있고 그 파일이 예컨대 butterfly.gif라면 url을 비우면 됩니다. 단 이 때는 웹에 접근하지 않을 것이므로 renewall 옵션이 작용하지 않도록 해두세요.
\begin{verbatim}
\animatedgifonthefly[width=\textwidth,autoplay,loop]%
        {12}{butterfly}{0}{5}{}
\end{verbatim}

\medskip

\cs{animatedgifonthefly}는 \cs{animategraphicsonthefly}라고 쓸 수도 있습니다.

\section{다른 명령의 인자가 될 때}

\marginpar{[v0.9]}
\cs{includegraphicsonthefly}와 \cs{animatedgifonthefly} 명령은 예컨대 figure 환경이나
minipage 안에 오는 경우 등에 대부분 정상적으로 동작합니다. 다만, \dotemph{다른 명령의 인자가 될 수 없습니다}.
예를 들어 kswrapfig 패키지가 제공하는 \cs{kwrapfig} 명령을 쓰려한다면,
\begin{verbatim}
\kswrapfig[UseBox=true,Pos=r]%
	{ \includegraphicsonthefly{testme}{<url>} }%
	{ <text> }
\end{verbatim}
이런 식으로 써야 하는데 이 경우 ``인자''로 들어갔기 때문에 오류가 발생합니다.

\begin{boxedverbatim}
\prepareimgonthefly<star>{파일이름}{url}
\end{boxedverbatim}
이럴 때는 두 단계로 작업합니다. 먼저 그림만 다운로드받고 그림의 삽입은 그 후에 하는 것. 그림을 다운로드하기만 하는 명령이 \cs{prepareimgonthefly}입니다. \verb|<star>|는 별표(*)를 줄 수도 있다는 뜻이고 파일이 이미 있더라도 갱신하라는 것입니다.

위의 오류가 발생하는 예제는 이제 다음처럼 할 수 있습니다.

\begin{verbatim}
\prepareimgonthefly{testme}{<url>}
\kswrapfig[UseBox=false,Pos=r]{testme}%
   {<text>}
\end{verbatim}

\begin{boxedverbatim}
\useimgonthefly[<option>]
\end{boxedverbatim}
이 명령은 직전에 다운로드받은 그림을 바로 사용합니다. 옵션은 \cs{includegraphics}에 넘겨줄 옵션입니다.
즉 \cs{prepareimgonthefly} 명령을 준 직후에 \cs{useimgonthefly}를 쓰는 것으로 파일 이름을 지정할 필요가 없습니다. (그럼 직전이 아니라 아까 다운받은 그림은? 그건 파일이름을 지정하였으니 그것으로 \cs{includegraphics}하면 되겠지요.)

위의 예제를 이 명령을 사용하여 써보면
\begin{verbatim}
\prepareimgonthefly{testme}{<url>}
\kswrapfig[UseBox=true,Pos=r]{\useimgonthefly[width=4cm]}
{<text>}
\end{verbatim}

\medskip

animated gif에 대해서도 비슷하게 할 수 있습니다. 일단 다운로드 명령은 \cs{prepareimgonthefly}로 동일합니다.

\begin{boxedverbatim}
\animatedgif[<option>]{<fps>}{파일이름}{start}{end}
\end{boxedverbatim}

만약 \cs{prepareimgonthefly}로 다운받은 파일이 animated gif라면 해당 파일 이름에 대하여 
위의 명령을 씁니다. url이 없는 \cs{animatedgifonthefly}인데 마지막 \verb|<end>|에 별표를 쓸 수
있습니다.

\begin{boxedverbatim}
\usegifonthefly[<option>]{fps,start,end}
\end{boxedverbatim}
\cs{useimgonthefly}와 같은 맥락에서 직전에 다운받은 파일을 사용하는 명령입니다.
\verb|<option>|은 \cs{animategraphics}에 넘겨줄 옵션이며 fps, start, end는 각각 위에서
설명한 바와 같습니다. \verb|end| 자리에 별표를 쓰면 36프레임까지 자동으로 포함합니다.

\bigskip

이 명령을 사용한 실제 예를 살펴보겠습니다.

\begin{verbatim}
\prepareimgonthefly{butt}%
   {http://www.animatedgif.net/animals/bugs/butrfly_e0.gif}
\InsertBoxL{2}{\usegifonthefly[loop,autoplay,width=3cm]{12,0,*}}
\jiwon[1]
\end{verbatim}

gif 그림을 \cs{InsertBoxL} 명령의 \dotemph{인자로 써야 하므로} \cs{animatedgifonthefly} 명령은 적절치 않아서 먼저 \cs{prepareimgonthefly}하여 그림을 준비하고 실제로 명령의 인자 안에서는 \cs{usegifonthefly}하였습니다. 만약 이것이 animated gif가 아니라 그냥 그림이라면 \cs{usegifonthefly} 대신 사용법이 더 간단한 \cs{useimgonthefly}를 사용하면 되겠습니다. 이들 \cs{use...} 명령은 \dotemph{직전에 다운받은} 그림을 사용한다는 것을 잊지 마세요.

위의 명령의 결과는 다음과 같습니다.

\begin{framed}
\prepareimgonthefly{butt}%
   {http://www.animatedgif.net/animals/bugs/butrfly_e0.gif}%
\InsertBoxL{2}{\usegifonthefly[loop,autoplay,width=3cm]{12,0,*}}%
\jiwon[1]
\end{framed}

\section{주의사항}

\begin{itemize}\firmlist
\item url이 잘못되어서 그림을 다운받을 수 없는 경우의 에러에 대해서는 책임지지 않습니다. url은 이미지 자체에 대한 링크여야 하고 이미지를 다운받게 하는 웹페이지로의 링크여서는 안됩니다.
\item Animate 효과는 Adobe Reader로 볼 때만 나타납니다.
\end{itemize}

\section{질문과 답변}
\hangfrom{문: }어떤 목적으로 사용하는 것인지요? 웹에 올라와 있는 그림이라면, 언제라도 변경될 수 있는 가능성이 있기에, 잘되면 컴파일 할때마다 최신의 화일을 얻는 것이겠지만, 잘 못되면, 그림이 빠진 문서가 생성될 가능성이 높지 않은지요? (likesam)

\hangfrom{답: }웹상의 그림을 문서에 포함한다고 할 때, 브라우저를 열고 $\rightarrow$ 다운로드받고 $\rightarrow$ 다운로드 폴더로 이동해서 그림을 작업중인 폴더로 가져오고 $\rightarrow$ 그것을 includegraphics하는데요, 이 과정을 좀 게으르게 가자는 목적입니다.
일단 컴파일에 성공하면 로컬 폴더에 해당 그림이 저장되는 것이고요. 매번 컴파일할 때마다 새로 다운받지 않습니다. 폴더에 해당 그림이 있으면 \cs{includegraphics}와 똑같습니다.
그리고 ``잘못되는 경우"는 생각할 필요 없지 않을까요? 그건 그 때 가서 문제고 어차피 url이 잘못되어 해당 그림을 다운받을 수 없는 것은 어째도 마찬가지일 테니까요.
Con\TeX{t}의 \cs{externalfigure} 명령의 \LaTeX\ 대응이라고 생각하고 만든 것입니다.
\cs{animategraphics}의 경우도 마찬가진데요, 그림을 다운받고 $\rightarrow$ convert 돌리고 $\rightarrow$ png 개수 확인하고 $\rightarrow$ animategraphics하는데요, 이 과정을 좀 간단하게 하고 싶어서 만든 것입니다.

\end{document}
