\documentclass[b5paper]{article}

\usepackage{hangulfontset}

\usepackage[hmargin=1in,vmargin=1in]{geometry}

\usepackage{kotex-sections}

\protected\def\cs#1{\texttt{\textbackslash #1}}
\protected\def\ct#1{\texttt{#1}}

\usepackage{kotex-logo}

\def\kotex{\koTeX}
\def\luatexko{\hologo{LuaTeX}-\ko}
\def\xetexko{\hologo{XeTeX}-\ko}

\begin{document}

\title{옛날 H\LaTeX\ 절표제, kotex-sections}
\author{Nova De Hi}
\date{2014/06/09}

\maketitle

\section{소개}

\kotex-{}utf와 \luatexko, \xetexko의 다른 점 중 하나가 절표제입니다.

\kotex-{}utf에서 \ct{[hangul]} 옵션을 주면 \cs{section}에 대하여 ``제 1 절''로 식자됩니다. \luatexko와 \xetexko는 그냥 ``1''이지요.

저 자신은 절에 대하여 굳이 ``제''와 ``절''을 붙일 필요가 없다고 생각합니다만 이걸 원하는 경우도 없지 않을 터라, H\LaTeX\ 이래 유구한 역사와 전통을 지닌 이 기능을 추가해봤습니다.

\section{사용법}

\subsection{kotex-sections.sty}

\begin{verbatim}
\usepackage{kotex-sections}
\end{verbatim}

\texttt{\textbackslash{}usepackage\{kotex\}} 이후에 와야 하고요, \kotex에 \ct{[hangul]} 옵션이 있든지 없든지 \ct{[hangul]}을 준 것과 비슷하게 동작합니다.

\subsection{\cs{kscntformat}}

\luatexko, \xetexko에서 삭제되어 있는 명령 \texttt{\textbackslash{}kscntformat}을 쓸 수 있습니다.

\begin{verbatim}
\renewcommand{\thechapter}{\Hnum{chapter}}  
\kscntformat{chapter}{}{~마당}  
\end{verbatim}

\subsection{제한}

oblivoir는 이미 이것과 비슷한 기능을 가지고 있으므로 이 패키지가 필요없고요, \kotex-{}utf에도 이 기능이 이미 있으니까 이 패키지를 요구하지 않습니다. article, book, report에서 절표제를 수식하는 별도의 패키지를 사용하지 않았을 때만 (비교적) 정상적으로 동작할 거라고 예상합니다.

\section{예제}

\begin{verbatim}
\documentclass{report}  
\usepackage[hangul]{kotex}
  
\renewcommand{\thechapter}{\Hnum{chapter}}  
\kscntformat{chapter}{}{~마당}  
  
\usepackage{jiwonlipsum}  
  
\begin{document}  
  
\chapter{일야에 구도하한 기록}  
 
\section{강물은}
\jiwon  
\end{document}  
\end{verbatim}

\end{document}
