\documentclass[a4paper,hcr]{oblivoir}

\usepackage[mono=false]{libertine}

\usepackage{jiwonlipsum}

\pagestyle{hangul}

\begin{document}

\title{\texttt{jiwonlipsum} 패키지}
\author{Nova De Hi and Progress}
\date{\today}

\maketitle

\begin{abstract}
\texttt{lipsum}이라는 패키지가 있습니다. 외국에서 출판이나 그래픽 디자인, 편집 디자인, 폰트 디자인 따위의 작업을 하면서 텍스트 시안(sample)을 작성할 때, 의미 없는 문장을 채워넣기 위해 사용하는 것으로 대략 이해하시면 됩니다.  \texttt{jiwonlipsum}은 \texttt{lipsum}의 한글판이라 생각하시면 됩니다. 연암 박지원 선생의 \ccnm{열하일기} 가운데 `하룻밤에 아홉 번 강을 건너다(一夜九渡河記)' 문장을 빌어왔습니다. 패키지는 Nova De Hi님이 작성하였고 이 매뉴얼은 Progress가 썼습니다.
\end{abstract}

\tableofcontents*

\section{사용하기}

\begin{boxedverbatim}
\jiwon
\end{boxedverbatim}

\jiwon

\section{부분적으로 사용하기}

\begin{boxedverbatim}
\jiwon[1-12]
\end{boxedverbatim}

\jiwon[1-12]

\begin{boxedverbatim}
\jiwon[13-20]
\end{boxedverbatim}

\jiwon[13-20]

\begin{boxedverbatim}
\jiwon[21-28]
\end{boxedverbatim}

\jiwon[21-28]

\begin{boxedverbatim}
\jiwon[15]
\end{boxedverbatim}

\jiwon[15]

\begin{boxedverbatim}
\jiwon[23]
\end{boxedverbatim}

\jiwon[23]

\section{nopar와 numbers}

\begin{itemize}
\item 패키지 옵션으로 \texttt{[nopar]}를 주면 문단 끝의 \verb|\par|가 붙지 않습니다.

\item \texttt{[numbers]} 옵션을 주면 문단에 번호가 붙습니다.
\end{itemize}

\end{document}
