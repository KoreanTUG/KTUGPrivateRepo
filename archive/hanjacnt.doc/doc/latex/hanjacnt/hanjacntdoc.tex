% arara: xelatex
% arara: xelatex
% arara: lmkclean
\documentclass[b5paper,nanum]{oblivoir}

\setkomainfont(NanumMyeongjo)[](HanaMinA)
\setkomonofont(Malgun Gothic)[](HanaMinA)

\usepackage{fapapersize}
\usefapapersize{*,*,1in,*,1in,*}

\def\cs#1{\texttt{\textbackslash #1}}
\def\ct#1{\texttt{#1}}
\def\pkg#1{\textsf{#1}}
\usepackage{kotex-logo}
\def\xetexko{\hologo{XeTeX}-\ko}
\def\luatexko{\hologo{LuaTeX}-\ko}
\protected\def\pdftex{\hologo{pdfTeX}}

\usepackage{ifpxltex}

\usepackage{hanjacnt}

\hypersetup{colorlinks}

\usepackage{tabto,multido}
\TabPositions{3cm,6cm,9cm,12cm,15cm}

\begin{document}

\title{hanjacnt, 한자/한글 숫자와 카운터}
\author{Nova De Hi}
\date{\HanjaToday, v2.7}

\maketitle

\begin{abstract}
한자와 한글 카운터 수식자를 제공하는 \pkg{hanjacnt}를 완전히 다시 쓰고
기능을 확장하였다. 숫자를 한자와 한글 표기로 표현한다.
\end{abstract}

\tableofcontents*

\section{소개}

원래의 \pkg{hanjacnt} 패키지\footnote{%
	\url{http://ktug.kldp.net/jsboard/read.php?table=ktugbd&no=4810&page=118}}%
는 1부터 999까지의 한자 및 한글 카운터를 제공하던 패키지였다.
이를 완전히 새로 설계하였다.

이전에 비하여 달라진 점은 다음과 같다.
\begin{itemize}\firmlist
\item 카운터 수식자(counter modifier)뿐 아니라 숫자 자체를 한자로 식자하는 명령을 도입하였다.
\item 한자 숫자 표시 방법을 개선하였다. 다양한 방법으로 표기되는 한자 숫자 표현 방법을 지원한다.
\item 한자 숫자를 식자할 폰트를 지정할 수 있다.
\item 한글 숫자의 만 단위로 띄어쓰기를 할 수 있다.
\item 큰 숫자도 잘 처리한다.
\end{itemize}

그러나 이전에 제공되던 두 개의 카운터 명령 \cs{HANJA}와 \cs{HANGUL}의 결과는 그대로이므로
이전에 작성한 문서에서도 문제를 일으키지 않을 것이다.

\section{사용법과 명령}

\begin{boxedverbatim}
\usepackage{hanjacnt}
\end{boxedverbatim}

패키지 옵션으로 \ct{[manspace]}와 \ct{[20]}, \ct{[finhanja]}를 줄 수 있다. \ct{[manspace]}는 \ref{sec:manspace}절에서, \ct{[20]}은 \ref{sec:twenty}절에서, \ct{[finhanja]}는 \ref{sec:fin}절에서 설명한다.

\subsection{숫자의 표현}

\begin{boxedverbatim}
\NumHanja{<number>},    \NumHangul{<number>}
\NumHanjaBig{<number>}, \NumHangulBig{<number>}
\end{boxedverbatim}

숫자를 한자 또는 한글로 표현한다.
예를 들면 다음과 같다.
\begin{verbatim}
\NumHanja{1}, \NumHanja{2}, \NumHanja{10}, \NumHanja{96} \\
\NumHangul{1}, \NumHangul{2}, \NumHangul{10}, \NumHangul{96}
\end{verbatim}
\IfpxlTeX*[F]{p}
{
\NumHanja{1}, \NumHanja{2}, \NumHanja{10}, \NumHanja{96} \\
\NumHangul{1}, \NumHangul{2}, \NumHangul{10}, \NumHangul{96}
}

\medskip

\cs{NumHanja}(\cs{NumHangul})로 사용 가능한 숫자의 범위는 10억 미만(아홉자리)이다. 
\cs{NumHanjaBig}(\cs{NumHangulBig})은 이보다 큰 숫자나 쉼표로 구분된 숫자를 표현하기 위한 명령이고, 
열세 자리(조)까지 숫자를 처리할 수 있다.
%\footnote{%
%	이 패키지에서 유효한 숫자는 쉼표로 자릿수를 구분하지 않는다.}

\begin{verbatim}
\NumHanjaBig{1205345624967}\\
\NumHangulBig{1205345624967}\\
\NumHanjaBig{2,345,100,124,287}\\
\NumHangulBig{2,345,100,124,287}
\end{verbatim}
\NumHanjaBig{1205345624967}\\
\NumHangulBig{1205345624967}\\
\NumHanjaBig{2,345,100,124,287}\\
\NumHangulBig{2,345,100,124,287}

\medskip

이 패키지는 내부적으로 다섯 자리까지는 `숫자'로 보고 처리한다.
그러나 여섯 자리부터는 숫자로 보는 것이 아니라 토큰열(sequence)로 취급한다.
[v2.2] 이전 버전에서 큰 숫자의 경우 직접 숫자가 아니라 control sequence로
넘어오는 경우 확장해야 했던 불편을 없앴다. 그러므로 다음과 같이 자연스럽게 쓸 수 있게 되었다.
%그래서 \cs{NumHanja}든 \cs{NumHanjaBig}이든 6자리 이상의 숫자가
%직접 숫자 자체가 아니라 control sequence로 넘어온다면 확장해주지 않으면 오류가 발생한다.
%다섯 자리까지는 그대로 써도 상관없다.
\begin{verbatim}
\def\test{20050}
\newcount\testb\testb=91234
\NumHanja{\test}, \NumHangul{\test},
\NumHanja{\the\testb}, \NumHangul{\the\testb} \\
\def\test{200500}\testb=1234567
\NumHanja{\test},
\NumHanja{\the\testb},
\NumHangul{\test},
\NumHangul{\the\testb}
\end{verbatim}

\def\test{20050}
\newcount\testb\testb=91234
\NumHanja{\test}, \NumHangul{\test},
\NumHanja{\the\testb}, \NumHangul{\the\testb}

\def\test{200500}\testb=1234567
\NumHanja{\test},
\NumHanja{\the\testb},
\NumHangul{\test},
\NumHangul{\the\testb}

\medskip

[v2.4] \cs{NumHanjaBig}과 \cs{NumHangulBig}은 큰 숫자를 다루기 때문에 쉼표로 세 자리마다 구분한 숫자도 처리한다.
주의할 것은 \cs{NumHanja}나 \cs{NumHangul} 및 \cs{NumHanjaDig}, \cs{NumHangulDig}은 쉼표로 구분된 숫자를 다루지 않는다는 것이다.

\begin{verbatim}
\renewcommand\test{1,000,200,530,244}
\NumHanjaBig{\test}\\
\NumHangulBig{\test}\\
\NumHanjaBig{12,000}\\
\NumHangulBig{12,350}\\
\end{verbatim}
\renewcommand\test{1,000,200,530,244}
\NumHanjaBig{\test}\\
\NumHangulBig{\test}\\
\NumHanjaBig{12,000}\\
\NumHangulBig{12,350}

\subsection{한자 갖은자}\label{sec:fin}

[v2.3] 한자 숫자 표기에 갖은자를 쓰는 경우가 있다. 주로 돈의 액수를 표시할 때 쓴다.
갖은자 한자 숫자를 식자하게 하려면 \ct{[finhanja]} 옵션을 준다. fin은 financial에서 온 것이다.

단순히 \ct{[finhanja]}만을 선언하면 다음과 같이 숫자가 식자된다.
\begin{verbatim}
\NumHanja{123456789}
\end{verbatim}
\FinHanjaMode
\NumHanja{123456789}

\begin{boxedverbatim}
\FinHanjaMode
\end{boxedverbatim}

갖은자 한자를 쓰는 방법이 몇 가지 있어서 이 명령을 마련했다. 인자 없이 쓰면 기본값을 쓰게 되고
\ct{full}, \ct{var}, \ct{fullvar}를 인자로 주어 갖은자 쓰는 방법을 선택할 수 있다.
\begin{verbatim}
\FinHanjaMode{full}
\NumHanja{987612345}
\FinHanjaMode{var}
\NumHanja{987612345}
\FinHanjaMode{fullvar}
\NumHanja{987612345}
\FinHanjaMode
\NumHanja{987612345}
\end{verbatim}
\FinHanjaMode{full}
\NumHanja{987612345} \\
\FinHanjaMode{var}
\NumHanja{987612345} \\
\FinHanjaMode{fullvar}
\NumHanja{987612345} \\
\FinHanjaMode
\NumHanja{987612345}

\medskip

이 명령에 \ct{on}이나 \ct{off}를 주어서 갖은자 한자 모드를 켜거나 끌 수 있다.
\begin{verbatim}
\FinHanjaMode{off}
\NumHanja{987612345}
\FinHanjaMode{on}
\NumHanja{987612345}
\end{verbatim}
\FinHanjaMode{off}
\NumHanja{987612345} \\
\FinHanjaMode{on}
\NumHanja{987612345}

\medskip

\cs{FinHanjaMode}와 \cs{FinHanjaMode\{on\}}은 거의 같은 의미이다.
\cs{usepackage}하면서 \ct{[fullfinhanja]}나 \ct{[varfinhanja]}를 선언하면 처음부터 해당 모드의 갖은자 한자로 식자한다.

\medskip

갖은자로 한자를 식자하는 명령은 \cs{NumHanja}, \cs{NumHanjaBig}이다.
큰 수나 쉼표로 자릿수가 구분된 숫자에는 \cs{...Big}을 쓴다.

\begin{verbatim}
\NumHanjaBig{1,130,150,471}
\end{verbatim}
\NumHanjaBig{1,130,150,471}

\FinHanjaMode{off}

\subsection{한글 숫자의 띄어쓰기}\label{sec:manspace}

한글 맞춤법에 ``수를 적을 적에는 만 단위로 띄어쓴다''는 규정이 있다. 이 규정을 살려서 큰 숫자를 만 단위로
띄어쓰게 하려면 패키지 옵션으로 \ct{[manspace]}를 지정한다. 한자에는 이것이 적용되지 않는다.

\ManSpaceOn
\begin{verbatim}
\NumHangulBig{2345100024287}
\end{verbatim}
\NumHangulBig{2345100024287}

\begin{boxedverbatim}
\ManSpaceOn, \ManSpaceOff
\end{boxedverbatim}
이 설정을 중간에 끄거나 켜는 명령이다.

\begin{verbatim}
\ManSpaceOff
\NumHangulBig{2345100024287}
\end{verbatim}
\ManSpaceOff
\NumHangulBig{2345100024287}

\subsection{자릿수별로 표시하기}\label{sec:twenty}
\begin{boxedverbatim}
\NumHanjaDig{<number>}
\HanjaZero{<char>}
\HanjaZeroFont{<font>}
\TwentyHanjaChar[2|20|30|21|31]
\end{boxedverbatim}
\HanjaZeroFont{HanaMinA}
한자 숫자를 사용할 때 예를 들어 45의 경우 \NumHanja{45}와 같이 읽는 것이 일반적이지만
가끔 \NumHanjaDig{45}로 표시하는 것이 좋을 경우가 있다.
% 예전 개역성경 간이국한문판에서 절(verse)을
%표시하기 위해 이런 식의 표기법을 사용하였다.
이 표기법에서 20부터 29까지의 숫자는 \NumHanjaDig{20},
\NumHanjaDig{21}과 같이 쓰지 않고 \TwentyHanjaChar[20]\NumHanjaDig{20}, \NumHanjaDig{21}과 같이
표기할 수 있다.

%이 표기법과 관련하여 몇 가지 알아둘 점이 있다.
\begin{itemize}\firmlist
\item $0$을 표기하는 데 사용되는 글자 \ct{[U+3007]}(상형문자 zero)가 빠져 있는 폰트가 있다.
나눔명조와 나눔고딕이 그렇다. 이런 폰트를 주 폰트로 쓰고 있다면 상형문자 zero를 표기할 때 사용할 폰트만 지정해줄 수 있다. \cs{HanjaZeroFont}를 쓴다.
\verb|\HanjaZeroFont{HCR Batang LVT}|
\item 폰트를 바꾸지 않고 상형문자 zero에 해당하는 글자 자체를 바꿀 수 있다. \verb|\HanjaZero{○}|.
\end{itemize}

\hanjafontspec{HanaMinA}
$20$에 해당하는 한자는 원래 廿(스물 입)이다(이 글자는 KS X 1001에 포함되지 않기 때문에 많은 한글 폰트에서 이 글자가 결락되어 있다). 그 이체자는 卄이고 이 글자를 포함하고 있는 폰트가 더 많다. 어떤 글자를 쓸 것인가를 다음과 같이 \cs{TwentyHanjaChar}의 옵션으로 \ct{[2]}, \ct{[20]}, \ct{[21]}을 주어서 선택할 수 있다.\footnote{%
	이 문서의 기본 폰트인 나눔 명조, 나눔 고딕에도 廿이 없는 까닭에 이 단락에서 한자 폰트를 HanaMinA로 하였다.}

\begin{verbatim}
\TwentyHanjaChar[2]\NumHanjaDig{20}
\TwentyHanjaChar[20]\NumHanjaDig{20}
\TwentyHanjaChar[21]\NumHanjaDig{20}
\end{verbatim}
\TwentyHanjaChar[2]\NumHanjaDig{20}
\TwentyHanjaChar[20]\NumHanjaDig{20}
\TwentyHanjaChar[21]\NumHanjaDig{20}

\medskip

[v2.5] 한편, 30에 해당하는 한자 卅(서른 삽)을 쓰는 경우도 있을 수 있다. 원한다면 이 글자를 선언할 수 있는데 \cs{TwentyHanjaChar}에 \ct{[30]}을 지시하면 된다. 이 옵션이 들어오면 20도 스물 입을 사용한다.
마찬가지로 \ct{[31]}을 선언하면 20에 해당하는 글자를 廿으로 만든다. 30에 해당하는 글자에는 변경이 없다.
\begin{verbatim}
\TwentyHanjaChar[30]
\NumHanjaDig{20}, \NumHanjaDig{30}\\
\TwentyHanjaChar[31]
\NumHanjaDig{20}, \NumHanjaDig{30}
\end{verbatim}
\TwentyHanjaChar[30]
\NumHanjaDig{20}, \NumHanjaDig{30}\\
\TwentyHanjaChar[31]
\NumHanjaDig{20}, \NumHanjaDig{30}

\medskip

패키지 옵션으로는 \ct{[2]}, \ct{[20]}, \ct{[30]}이 있고 의미는 위에 설명한 바와 같다. \ct{[21]}이나
\ct{[31]}은 패키지 옵션이 아니라 \cs{TwentyHanjaChar} 명령으로 활성화한다.

\bigskip

[v2.6] 사실 스물 입이나 서른 삽과 같은 글자를 사용하여 숫자를 표시하는 일은 우리 문헌에 극히 드물다.
출판된 서적 가운데는 아마도 대한성서공회의 \bnm{성경전서} 개역한글판(1962)과 간이국한문(1964)이 유일한 예일 것이다.
이 책의 `관주'라 불리는 상호참조 정보에 쓰인 한자 숫자는 본문 장절표시를 위한 한자 숫자 체계와 또 달라서 매우 독특한 
표기를 보여주는데, 다음은 이를 재현해본 것이다. 이 표기에서 ○은 숫자 영($0$)이 아니고 장과 절 사이의 구분 기호이고 숫자 앞에 붙은 글자들은 성경의 책 이름 약자이다. 이런 것이 실제로 쓰일 일은 거의 없을 것이다.\footnote{%
	이 명령들은 스물 입과 서른 삽 글자가 갖추어진 폰트가 아니라면 제대로 나타나지 않는다.}
\begin{verbatim}
\KRVcom{22},
\KRVverse{Gen}{45.22}{25},
\KRVverse{1Ki}{13.34}
\end{verbatim}
\KRVcom{22},
\KRVverse{Gen}{45.22}{25},
\KRVverse{1Ki}{13.34}

\subsection{한자 숫자의 교체}
\begin{boxedverbatim}
\MarkHanja{ <name> } { <char> }
\end{boxedverbatim}
[v2.7] 각 숫자에 해당하는 한자를 사용자가 별도로 정의하여 쓸 수 있다.
이 명령은 두 개의 인자를 받아들이는데 첫 번째 것은 숫자의 이름이다. 이 자리에 올 수 있는 것은,
1부터 9까지의 숫자, one, two, three, four, five, six, seven, eight, nine과
십, 백, 천, 만, 억, 조에 해당하는 shi, bae, chu, man, eok, cho이다. 십, 백, 천은 ten, hundred, thousand를
쓸 수도 있다. 사용자 정의 한자를 식자하는 것은 갖은자 모드가 아닐 때이고 \cs{NumHanja}, \cs{NumHanjaBig}뿐 아니라 \cs{NumHanjaDig} 및 카운터 \cs{HANJA}와 \cs{HANJADIG}에도 효력을 미친다.
\begin{verbatim}
\MarkHanja{man}{万}
\MarkHanja{one}{壱}
\MarkHanja{two}{弍}
\NumHanjaBig{12,011}
\end{verbatim}
\MarkHanja{man}{万}
\MarkHanja{one}{壱}
\MarkHanja{two}{弍}
\NumHanjaBig{12,011}

\medskip

모두 기본값으로 되돌리려면 \cs{MarkHanja\{default\}}라고 선언한다. 이 경우 두 번째 인자는 없다.

\begin{verbatim}
\MarkHanja{default}
\NumHanjaBig{12,011}
\end{verbatim}
\MarkHanja{default}
\NumHanjaBig{12,011}

\medskip

이 기능을 이용하면 기본적으로 제공되지 않는 ``만 단위로 띄어쓰기''를 한자 숫자에도 적용할 수 있다.
\begin{verbatim}
\MarkHanja{eok}{億\space}
\MarkHanja{man}{萬\space}
\NumHanjaBig{123,453,012}
\end{verbatim}
\MarkHanja{eok}{億\space}
\MarkHanja{man}{萬\space}
\NumHanjaBig{123,453,012}

\MarkHanja{default}

\subsection{카운터}
\begin{boxedverbatim}
\HANJA{<counter>}, \HANGUL{<counter>}, \HANJADIG{<counter>}
\end{boxedverbatim}
위에 설명된 \cs{NumHanja}, \cs{NumHangul}, \cs{NumHanjaDig}과 같으나 숫자를 인자로 취하는 것이 아니라
카운터 변수를 취한다. 원래의 \pkg{hanjacnt}가 이러한 카운터 수식명령만을 제공하였다.

\begin{verbatim}
\newcounter{foo}
\stepcounter{foo}\HANJA{foo},
\addtocounter{foo}{5}\HANGUL{foo},
\setcounter{foo}{105}\HANJA{foo} \HANGUL{foo} \HANJADIG{foo}
\end{verbatim}
\newcounter{foo}
\stepcounter{foo}\HANJA{foo},
\addtocounter{foo}{5}\HANGUL{foo},
\setcounter{foo}{105}\HANJA{foo} \HANGUL{foo} \HANJADIG{foo}

\medskip

\cs{pagenumbering}과도 함께 쓸 수 있다. \verb|\pagenumbering{HANJA}|.
그러나 \pkg{memoir}나 \pkg{oblivoir}라면 다음과 같이 pagestyle을 정의하는 쪽을 권장한다.
\begin{verbatim}
\makeoddfoot{plain}{}{\HANJA{page}}{}
\end{verbatim}
\makeoddfoot{plain}{}{\HANJA{page}}{}
이제 이 페이지부터 plain 페이지스타일에서 바닥에 페이지 번호가 한자로 찍힐 것이다.
카운터로 다룰 수 있는 한자/한글 숫자의 범위는 10만 미만(다섯 자리)이다.
\begin{verbatim}
\setcounter{foo}{11111}
\HANJA{foo} \HANGUL{foo}
\end{verbatim}
\setcounter{foo}{11111}
\HANJA{foo} \HANGUL{foo}

\bigskip

갖은자 한자 카운터는 \cs{FinHanjaMode}에 의해 활성화된다. \cs{HANJADIG}으로는 갖은자를 식자할
필요가 없으므로 이와 무관하다.
\begin{verbatim}
\FinHanjaMode{on}
\HANJA{foo}
\end{verbatim}
\FinHanjaMode{on}
\HANJA{foo}

\FinHanjaMode{off}

\subsection{한글/한자 날짜}

날짜를 표시하는 데 한글이나 한자를 써야 하는 경우가 있다.

\begin{boxedverbatim}
\HanjaYear{ year }, \HanjaMonth{ month }, \HanjaDay { day },
\HangulYear{ year }, \HangulMonth{ month }, \HangulDay { day },
\HanjaToday, \HangulToday
\end{boxedverbatim}

\cs{HanjaToday}와 \cs{HangulToday}의 결과는 각각 다음과 같다.

\medskip

\HanjaToday, \HangulToday

\medskip

\cs{HangulLunarToday}라는 명령을 주면 표현 방식이 다음처럼 나타난다. 한자에는 해당 명령이 없다.
단, 이 패키지는 양력-음력 변환을 해주지는 않으므로 양력 날짜를 `음력식' 읽기로 보여주는 것뿐이다.

\medskip

\HangulLunarToday

\medskip

\cs{Hangul|Hanja||Year|Month|Day} 명령에는 인자로 숫자를 줄 수 있다. 한자 연도표시는 \cs{...Dig} 방식을 따른다. 일반적인 한자 읽기 방식을 쓰려면 직접 `\verb|\NumHanja{2002}年|'과 같이 쓰도록 하라 (\NumHanja{2002}年).

\begin{verbatim}
\HangulYear{2000}, \HanjaYear{2002} \\
\HangulMonth{10}, \HangulMonth{6}, \HanjaMonth{1} \\
\HangulDay{10}, \HanjaDay{15}
\end{verbatim}
\HangulYear{2000}, \HanjaYear{2002} \\
\HangulMonth{10}, \HangulMonth{6}, \HanjaMonth{1} \\
\HangulDay{10}, \HanjaDay{15}

\medskip

한글 날짜의 `음력식' 읽기는 \cs{HangulLunarDay}를 이용한다. 양력 날짜를 음력식으로 읽는 것이라서 (원래는 없는) 30일과 31일이 추가되어 있다. 만약 그 달의 마지막 날을 표시하려면 인자로 100을 준다.

\begin{verbatim}
\HangulLunarDay{15}, \HangulLunarDay{5}, \HangulLunarDay{100}
\end{verbatim}
\HangulLunarDay{15}, \HangulLunarDay{5}, \HangulLunarDay{100}

\medskip

이 명령을 인자 없이 쓰면 오늘 날짜를 음력식으로 읽는다. \cs{HangulLunarDay} \HangulLunarDay.
\ct{HanjaLunarDay}는 정의하지 않았다.

단기 연호를 사용하려면 \cs{HangulDangiYear}를 쓰면 된다. 이 명령은 인자를 반드시 주어야 하고 인자는 서기 연도여야 한다. \cs{HanjaDangiYear} 명령도 있다.
\begin{verbatim}
\HangulDangiYear{\the\year}, \HangulDangiYear{2007},
\HanjaDangiYear{\the\year}
\end{verbatim}
\HangulDangiYear{\the\year}, \HangulDangiYear{2007}, \HanjaDangiYear{\the\year}

\medskip

불멸기원도 마찬가지로 표현할 수 있다.
\begin{verbatim}
\HangulBulgiYear{\the\year},
\HanjaBulgiYear{\the\year}
\end{verbatim}
\HangulBulgiYear{\the\year},
\HanjaBulgiYear{\the\year}


\subsection{폰트}
\begin{boxedverbatim}
\NumHanjaFont{ <font> }
\end{boxedverbatim}
한자 숫자에 대해서만 별도의 폰트를 써야 할 때가 있다. 페이지번호 장식과 같은 것이 대표적일 것이다.
이 때 \cs{NumHanjaFont}를 쓰면 된다. 이것은 \cs{hanjafontspec} 명령을 이용하기 때문에
\xetexko, \luatexko가 아니면 의미가 없다. 여기서 폰트를 선언하면 그 효과는 이 페이지의 페이지 번호에 바로 나타난다.
\begin{verbatim}
\NumHanjaFont{HYhaeseo}
\end{verbatim}
\NumHanjaFont{HYhaeseo}

\subsection{\texorpdfstring{\pdftex}{pdfTeX}의 경우}
\pdftex에서 \koTeX-utf를 쓰는 상황이라면 이 패키지의 몇 가지 기능을 사용할 수 없다. \cs{HANJADIG}은
오류를 일으키지 않으나 상형문자 zero가 제대로 나타나지 않는다. 폰트 관련 명령은 \pdftex에서는 의미가 없다.
그러므로 \cs{HANJA}, \cs{HANGUL}, \cs{NumHanja}, \cs{NumHangul}을 쓸 수 있다고 생각하면 된다.


\section{바뀐 점}

\noindent\textbullet\ v2.7 (2015/06/28) 사용자 한자 숫자 정의 도입. 날짜표기.

\noindent\textbullet\ v2.5 (2015/06/27) 스물 입, 서른 삽의 처리. 스물 입의 이체자. 

\noindent\textbullet\ v2.4 (2015/06/26) 쉼표로 구분된 숫자의 처리.

\noindent\textbullet\ v2.3 (2015/06/26) 한자 갖은자 숫자의 식자 가능. (suggested by nomos)

\noindent\textbullet\ v2.2 (2015/06/26) \cs{expandafter}하던 문제를 해결.

\clearpage

\section{테스트}

한자 숫자 글꼴을 HanaMinA로 하여 테스트한다. 20은 \ct{[21]} 상태이다.
\TwentyHanjaChar[21]

\bigskip

\NumHanjaFont{HanaMinA}

\parindent=0pt
\long\def\TestLine{%
	\stepcounter{foo}\thefoo \tab
	\HANGUL{foo} \tab
	\HANJA{foo} \tab
	\HANJADIG{foo} \par
}

\setcounter{foo}{-1}
카운터\tab \cs{HANGUL}\tab \cs{HANJA}\tab \cs{HANJADIG} \\
\multido{\i=1+1}{51}{ \TestLine }

\setcounter{foo}{80}
\multido{\i=1+1}{30}{ \TestLine }

\setcounter{foo}{1000}
\thefoo\tab \HANGUL{foo}\tab \HANJA{foo}\tab \HANJADIG{foo} \par

\setcounter{foo}{\the\year}
\thefoo\tab \HANGUL{foo}\tab \HANJA{foo}\tab \HANJADIG{foo} \par


\end{document}
